\subsection{150 Ore}
Tra le opportunità previste dalla legge n.390 del 1991 a sostegno del diritto allo studio, l'università ha previsto le cosiddette 150 ore ovvero collaborazioni con gli studenti per lo svolgimento di diverse attività (tutor dei laboratori informatici, supporto alle biblioteche o alle Segreterie Studenti, orientamento per le matricole, etc.) per un massimo – appunto - di 150 ore. Queste collaborazioni sono a tempo parziale e retribuite 9€ all'ora. Si accede tramite bando basato su criteri di reddito e merito, per cui è necessario essere iscritti almeno al secondo anno per potervi partecipare. Durante ogni anno accademico vengono proposti diversi bandi, sia d'Ateneo sia di Facoltà, per i quali possono essere richiesti alcuni requisiti tecnici (es. abilità informatiche di base). Se siete interessati, controllate periodicamente http://www.unimib.it/go/45056 sul quale vengono pubblicati bandi e graduatorie.

\subsection{Job Placement}
Promosso dall'ufficio Job Placement, il servizio VULCANO (Vetrina Universitaria Laureati con Curricula per le Aziende Navigabile On-line) offre la possibilità a tutti gli studenti, laureati o laureandi, di essere inseriti in un database di curricula che permette all'ufficio di favorire e perseguire l'incontro tra offerta e domanda di lavoro. Gli iscritti al servizio ricevono via mail proposte inoltrate da aziende interessate ai profili professionali proposti, sia per offerte di lavoro sia per possibilità di stage.
Ogni anno, inoltre, l'Ateneo organizza il Career Day, giornata d'incontro tra studenti/laureati e aziende dove vengono proposti anche attività per la stesura di un buon curriculum e per l'introduzione al mondo del lavoro.
Per maggiori informazioni, è utile consultare la pagina web http://http://www.unimib.it/go/45763, dove sono presenti anche gli orari dello Sportello d'Orientamento.
 
\subsection{Counselling psicologico}
La Bicocca offre anche un utile, ma spesso sconosciuto, servizio di counselling per chi avesse problemi di studio o situazioni personali che inibiscono il corretto svolgimento della carriera universitaria. Il servizio consiste in un ciclo medio-breve di incontri individuali con uno psicoterapeuta od uno psicologo clinico specializzati ad operare con pazienti in età tardo- adolescenziale e giovane-adulta. Durante il percorso lo specialista cercherà di stabilire con lo studente degli obiettivi chiari da raggiungere entro la fine della serie di incontri.
Il servizio è totalmente gratuito, per prenotare un incontro o semplicemente avere informazioni è possibile inviare un' e-mail o prendere contatto telefonicamente. I contatti sono disponibili all'indirizzo http://www.unimib.it/go/46063.

\subsection{Sport}
Correre per arrivare in università o per prendere il treno all'ultimo non si può considerare propriamente uno sport. Meglio qualcosa di classico, divertente e, possibilmente, a buon prezzo.
Per questo potete rivolgervi al CUS (Centro Universitario Sportivo), che offre corsi a prezzi convenienti, fornisce sconti ed agevolazioni per alcuni impianti comunali e organizza settimane bianche, gite ed escursioni. In Bicocca c'è una palestra convenzionata (in U12), accessibile a studenti e dipendenti, aperta dal lunedì a venerdì dalle ore 12 alle ore 20 (dalle ore 10 alle ore 12 solo per i residenti del pensionato) che ha dei prezzi veramente imbattibili! Per informazioni www.cusmilano.it, su Facebook http://www.facebook.com/cusbicocca, oppure rivolgetevi al CUS point in U6 al primo piano.

\subsection{Biblioteche e aree studio}
La Biblioteca d'Ateneo ha un'unica gestione ma è strutturata in tre sedi (quella Centrale, di Scienze e di Medicina). La biblioteca è il luogo ideale per studiare in tranquillità e che offre molti servizi: consultazione, prestito, prestito interbibliotecario, fotocopie, reperimento di articoli di vari natura, consulenza bibliografica, spazi di studio individuali per chi è sotto tesi e poi vi è un intuitivo catalogo on line (OPAC).
Le tre sedi sono:
\begin{itemize}
\item Sede Centrale: Piazza dell'Ateneo Nuovo 1, edificio U6 II piano. Aree disciplinari: diritto, economia, informatica, psicologia, sociologia, scienze della formazione e statistica. Orario di apertura: dal lunedì al giovedì dalle 9.00 alle 19.30 e il venerdì dalle 9.00 alle 18.30.
\item Sede di Scienze: Piazza della Scienza 3, edificio U2 I piano (sala monografie, dal lunedì al venerdì, dalle 9.00 alle 18.30) e piano -1 (sala periodici, aperta fino alle 16.00). Aree disciplinari: matematica, fisica, biologia, chimica e geologia.
\item Sede di Medicina: Via Cadore 48, Monza, edificio U8 piano terra. Aree disciplinari: medicina. Orario di apertura: dal lunedì al venerdì, dalle 9.00 alle 18.30.
\end{itemize}
Inoltre nel campus della Bicocca è presente un'altra Biblioteca afferente al CIDiS, che si trova al secondo piano dell'edificio U12. A differenza della biblioteca d'Ateneo è gestita da un organo interunivesitario (appunto il CIDiS) . Per poter usufruire del prestito dei libri è necessario pagare una quota di 10 euro. Grazie al successo della sperimentazione degli anni scorsi sull'apertura serale, la biblioteca del CIDiS ha confermato gli orari di apertura dal lunedì al venerdì dalle 9.00 alle 22.00 ed ha esteso i suoi servizi alla serata del sabato dalle 18.00 alle 22.00.
Altri luoghi di studio sono le Aree Studio sparse tra i vari edifici dell'Università, dove si può trovare una maggiore tranquillità rispetto ai tavoli nei corridoi o nei cortili interni.

\subsection{Copisterie}
Quasi certamente nella vostra carriera universitaria vi troverete con la necessità di stampare dispense in media da 200 pagine, fotocopiare una lezione persa oppure un intero quaderno di appunti di una vostra compagna, per non parlare poi della tesi. Le principali copisterie nelle vicinanze di Piazza della Scienza e di Piazza dell'Ateneo Nuovo sono:
\begin{itemize}
\item Copisteria al n°7: via Luigi Pulci, 7;
\item All.net: p.zza della Trivulziana, 2;
\item Digicopy: viale Sarca, 173;
\item Centro Copie Bicocca: viale Sarca, 198.
\end{itemize}
