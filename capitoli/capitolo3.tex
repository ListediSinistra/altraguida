\mysec{Le segreterie}
In università ci sono due segreterie principali, a cui noi studenti dobbiamo rivolgerci:
\begin{itemize}
\item la segreteria studenti è l'ufficio responsabile dello status dello studente rispetto all'università. Questa segreteria si occupa di pratiche come le immatricolazioni, la consegna delle domande di laurea, il trasferimento da o verso un'altra università o un altro corso di laurea, l'interruzione degli studi...
\item la segreteria didattica è invece l'ufficio responsabile dell'organizzazione della didattica per ogni corso di laurea. Probabilmente avrete molto a che fare con questa segreteria, che si occupa di questioni più quotidiane nella vita dello studente come gestire gli aspetti burocratici degli esami, stabilire l'orario delle lezioni, compilare i piani di studi, gestire la burocrazia dei tirocini e delle tesi di laurea.
\end{itemize}

\subsection{Segreterie On Line}
Tutte le pratiche burocratiche vengono gestite tramite ESSE3, il sistema telematico che l'Università utilizza per gestire le carriere accademiche di tutti gli studenti. 
Ogni studente ha un profilo personale a cui può accedere da qualsiasi computer tramite internet o dai terminali blu che si trovano in giro per l'università. Lo scopo è quello di velocizzare molti aspetti burocratici; questo sistema permette infatti di stampare documenti (come autocertificazioni riguardo la propria carriera accademica o i bollettini MAV per i pagamenti delle tasse), di compilare il proprio piano di studi, di partecipare ai bandi indetti dall'Università, di fare domanda di laurea. L'utilizzo principale delle segreterie online è per l'iscrizione agli esami e per visualizzare il proprio libretto con voti, medie, crediti acquisiti. 
Il sistema è migliorato negli ultimi anni, ma ancora migliorabile; state attenti ed evitate di ridurvi all'ultimo minuto, se ci fossero dei problemi e non possiate completare le vostre pratiche, la segreteria didattica dovrebbe potervi aiutare.
Grazie al nuovo metodo di verbalizzazione online, dopo aver effettuato l'esame, appena il professore chiude il registro i voti vengono caricati automaticamente sulla vostra pagina personale. In questo modo le carriere degli studenti sono aggiornate immediatamente con voti e crediti acquisiti.

\subsubsection{Immatricolazione}
Anche la procedura di immatricolazione avviene attraverso il sistema ESSE3, seguendo la procedura indicata dalla voce "registrazione". Il tutto avviene completamente online, senza che sia necessario presentarsi in segreteria o spedire documenti. Per completare correttamente la propria iscrizione è necessario caricare una foto (formato fototessera, un .jpeg 660x791 a 72 DPI è il formato consigliato), compilare tutti i campi richiesti e infine stampare il MAV e pagare la prima rata.
La procedura è leggermente diversa per alcuni corsi di laurea (ad esempio se sono a numero programmato, se c'è un test di ammissione) ma è accuratamente spiegata sul sito dell'università alla voce immatricolazione (http://www.unimib.it/go/45470).\\
Per avere altri dettagli potete sempre consultate la Guida dello Studente (scaricabile dal sito dell'università) o contattare il servizio di Orientamento d'Ateneo ( www.unimib.it/orientamento o con una mail a orientamento@unimib.it).
