5.2 Servizi per la disabilità 
Come previsto dalla legge n. 17 del 1999, in tutti gli atenei è stato creato un ufficio adibito ad offrire servizi a studenti disabili. Quello della nostra università si chiama Ufficio DAB (Diversamente Abili Bicocca) ed è nato con lo scopo sia di favorire l'integrazione sociale degli studenti disabili sia di facilitarne la permanenza all'interno dell'università stessa. 
I servizi previsti corrispondono sia a sussidi tecnici e didattici, come la registrazione dei corsi, l'adattamento dei libri di testo e tutoring, sia a mezzi di trasporto specifici e accompagnamento in università e nelle sue pertinenze. Per poter usufruire dei servizi erogati dall'ufficio, è necessario presentare all'inizio dell'anno accademico (dal 1 settembre al 30 settembre) una domanda di iscrizione all'ufficio reperibile presso la Sede Operativa, in seguito sarà fatto un colloquio con lo staff per stilare insieme al candidato un piano personalizzato dei servizi. 
La Sede Operativa è situata nell'edificio U6 a piano terra. Altrimenti si può contattare il DAB tramite mail  servizi.disabili@unimib.it, o ai seguenti numeri: 02.6448.6981/6984 - 02.6448.6067 
Ulteriori informazioni sono disponibili sul sito dell'università http://www.unimib.it/go/126829762.
