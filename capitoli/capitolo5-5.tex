5.11 Erasmus: studiare all'estero
Anzitutto, che cos'è il Progetto Erasmus? L'Erasmus è il principale progetto europeo di mobilità studentesca internazionale e permette ogni anno a migliaia di studenti di muoversi per un periodo di tempo (dai tre mesi ad un anno) ed andare a studiare e dare esami in un altro paese europeo. Lo studente in Erasmus è equiparato agli studenti dell'università ospitante e avrà quindi accesso a tutti i servizi offerti loro, oltre ad alcuni servizi specifici come l'aiuto nella ricerca dell'alloggio o i corsi di lingua. Per partire è necessario consultare i bandi che ogni anno la propria facoltà pubblica, solitamente in febbraio/marzo dell'anno precedente a quello che vi interessa, e fare domanda attraverso gli appositi uffici (la procedura completa è descritta all'interno del bando di riferimento). 
Ogni facoltà propone le destinazioni con cui è convenzionata. Da due anni grazie al lavoro dei nostri rappresentanti in CdA la Bicocca stanzia i fondi per le borse prima dell'inizio dei periodi di studio all'estero, quindi possiamo già dirvi che chi partirà per l'Erasmus nell'A.A.12/13 riceverà 300€ ogni mese. Proveremo l'anno prossimo a chiedere alla Bicocca di anticipare ancora la delibera in modo che si sappia già a febbraio (quando vanno presentate le richieste) a quanto ammonterà la borsa, che in ogni caso non dovrebbe ridursi rispetto agli anni precedenti. Fino allo scorso anno si aspettava lo stanziamento ministeriale e si potevano approvare le borse solamente a Febbraio/Marzo, cioè un anno dopo il termine per presentare le domande e quando molti studenti erano già partiti. Anche il CIDiS (Consorzio Interuniversitario per il Diritto allo Studio) bandisce delle borse per gli studenti in partenza per università estere assegnate in base al reddito e ai punti di credito conseguiti negli anni. Possono partire tutti gli studenti iscritti ad un anno successivo al primo (il primo anno si può fare domanda per partire il secondo). Prima di partire è necessario stipulare un learning agreement, ovvero un elenco degli esami che si intendono sostenere all'estero, per avere la certezza, al ritorno, che tutti gli esami sostenuti siano riconosciuti e quali saranno, per esempio, i crediti e il voto attribuiti ad ognuno (non tutti i paesi usano sistemi con crediti e voti in trentesimi).

Lingue
L'università di partenza può richiedere la conoscenza della lingua del paese di destinazione che avete scelto, possibilmente certificata da un diploma. Altrimenti è comunque sufficiente aver passato il test di conoscenze linguistiche di inglese (o della lingua del paese di destinazione) di Ateneo. Di solito, inoltre, per le lingue meno conosciute, le Università di destinazione organizzano dei corsi specifici destinati agli studenti Erasmus.
Per un contatto diretto con l'ufficio della Bicocca che si occupa degli Erasmus, scrivete a: international.office@unimib.it

5.12 Altri programmi di scambio
Erasmus mundus
Interessante anche questo progetto, che offre la possibilità di studiare all'estero, non durante il proprio percorso accademico curricolare, bensì attraverso un master. 

Extra
Nuovo programma dell'Università degli Studi di Milano-Bicocca di mobilità che si rivolge a chi ha quasi concluso il proprio percorso di studi. Consente lo svolgimento di un periodo di studio all'estero finalizzato alla preparazione della tesi di Laurea Specialistica/Magistrale, della durata minima di 3 mesi e massima di 6, presso università o centri di ricerca con i quali siano già attivi contatti e/o iniziative di collaborazione accademica o scientifica con l'Università Bicocca. 
Il bando richiede la presentazione delle domande in tre scadenze quadrimestrali, attraverso le quali illustrare il proprio progetto. Nell'ambito di tale Programma, sono stati finanziati dei premi di studio da parte della Fondazione Cariplo dell'importo mensile lordo pari a 750€. 

Exchange
Con il Programma Exchange è possibile fare un'esperienza di studio in paesi europei ed extra europei presso uno dei Partners Exchange di Ateneo, per un periodo che può andare da un minimo di alcune settimane ad un anno, durante il quale studiare e dare esami che saranno riconosciuti nel piano di studi ai fini della laurea. L'Exchange è un'ulteriore possibilità di mobilità a cui può partecipare anche chi ha già fatto l'esperienza dell'Erasmus. I bandi Exchange escono verso marzo. 

Summer school
Le Summer Schools offrono la possibilità di andare in un'università estera per frequentare corsi estivi di approfondimento su tematiche o settori del proprio corso di laurea, della durata di alcune settimane. I bandi di partecipazione si possono trovare sul sito di ateneo ma anche (e soprattutto) sui siti delle facoltà che li organizzano. I corsi seguiti durante la Summer School possono essere riconosciuti come CFU a scelta, ma solo su richiesta dello studente e comunque in questo caso non è previsto un accordo tra l'università di appartenenza e quella straniera. 
