\mysec{Mobilità internazionale}
\subsection{Erasmus: studiare all'estero}
Anzitutto, che cos'è il Progetto Erasmus? L'Erasmus è il principale progetto europeo di mobilità studentesca internazionale e permette ogni anno a migliaia di studenti di muoversi per un periodo di tempo (dai tre mesi ad un anno) ed andare a studiare e dare esami in un altro paese europeo, oppure svolgere uno stage lavorativo.\\
Lo studente in Erasmus per studio è equiparato agli studenti dell'università ospitante e avrà quindi accesso a tutti i servizi offerti loro, oltre ad alcuni servizi specifici come l'aiuto nella ricerca dell'alloggio o i corsi di lingua. Per partire è necessario consultare i bandi che ogni anno la propria facoltà pubblica, solitamente in febbraio/marzo dell'anno precedente a quello che vi interessa, e fare domanda attraverso gli appositi uffici (la procedura completa è descritta all'interno del bando di riferimento). \\
Ogni facoltà propone le destinazioni con cui è convenzionata.\\
Possono partire tutti gli studenti iscritti ad un anno successivo al primo (il primo anno si può fare domanda per partire il secondo). Prima di partire è necessario stipulare un learning agreement, ovvero un elenco degli esami che si intendono sostenere all'estero, per avere la certezza, al ritorno, che tutti gli esami sostenuti siano riconosciuti e quali saranno, per esempio, i crediti e il voto attribuiti ad ognuno (non tutti i paesi usano sistemi con crediti e voti in trentesimi).\\
Lo studente in Erasmus per studio riceve un contributo mensile per coprire parte delle spese del soggiorno all'estero. Esso è erogato dall'Ateneo su finanziamenti europei e può essere di 230€ o di 280€ per ogni mese di permanenza all'estero, in base al costo della vita nel Paese dell'università di destinazione. Ad esso va sommata un'integrazione che la Bicocca fornisce con fondi propri, la quale viene decisa ogni anno e la cui entità dipende dall'ISEEU dello studente.\\
A tal proposito, lo studente che intende andare in Erasmus per studio dovrebbe ricordarsi di presentare la dichiarazione ISEEU a settembre dell'anno accademico in cui partecipa al bando (quindi l'anno prima di quello in cui effettivamente parte). In caso contrario non potrà ricevere l'integrazione.\\
Anche il CIDiS (Consorzio Interuniversitario per il Diritto allo Studio) bandisce delle borse per gli studenti in partenza per università estere assegnate in base al reddito e ai punti di credito conseguiti negli anni.\\
Gli studenti interessati ad un'esperienza di stage all'estero devono invece partecipare ai bandi di Erasmus Placement. In questo caso la scelta della destinazione non è vincolata ad accordi tra la Bicocca e altre università ma dipende dall'ente in cui si intende svolgere lo stage, che può essere un'azienda o un centro di ricerca e formazione. La ricerca di un ente ospite spetta allo studente, il quale però può provare a chiedere suggerimenti ai propri professori.\\
La borsa di studio per l'Erasmus Palcement ammonta a 500€ al mese.\\
Quante volte si può andare in Erasmus? Da quest'anno, con il nuovo programma Erasmus+, ogni studente ha di diritto fino a 12 mesi di permanenza all'estero, anche non consecutivi, per ciclo di studi. Questo vuol dire che triennale e magistrale vengono contate separatamente: passando alla magistrale i mesi a disposizione tornano ad essere 12!

\subsubsection{Lingue}
Se si va in Erasmus per studio, l'università di partenza può richiedere la conoscenza della lingua del paese di destinazione che avete scelto, possibilmente certificata da un diploma. Altrimenti è comunque sufficiente aver passato il test di conoscenze linguistiche di inglese (o della lingua del paese di destinazione) di Ateneo. Di solito, inoltre, per le lingue meno conosciute, le Università di destinazione organizzano dei corsi specifici destinati agli studenti Erasmus.\\
Da quest'anno, tuttavia, la Bicocca richiede una competenza linguistica di inglese o della lingua dell'ateneo ospite, se è tra le lingue europee più parlate, almeno di livello B2 (quindi più alta di quella del test iniziale di ateneo) prima di acconsentire alla partenza. Chi non dispone di tale certificato dovrà seguire dei corsi di lingua e superare un test finale che si terrà a Luglio.


\subsection{Altri programmi di scambio}
La Bicocca offre poi altri programmi di scambio:

\begin{itemize}
\item Erasmus Mundus: progetto che offre la possibilità di studiare all'estero, non durante il proprio percorso accademico curricolare, bensì attraverso un master. 

\item Extra: progetto che consente lo svolgimento di un periodo di studio all'estero finalizzato alla preparazione della tesi di Laurea Specialistica/Magistrale presso università o centri di ricerca con i quali siano già attivi contatti e/o iniziative di collaborazione accademica o scientifica con l'Università Bicocca. Nell'ambito di tale Programma, sono stati finanziati dei premi di studio da parte della Fondazione Cariplo dell'importo mensile lordo pari a 750€.\\
Nota bene: da quest'anno si può accedere al bando Extra solo se si ha già avuto un'esperienza di studio all'estero. Quindi se vuoi parteciparvi ti conviene pianificare di andare in Erasmus l'anno prima, o anche l'anno stesso.

\item Exchange: programma che da la possibilità di studiare e dare esami in un paese europeo o extra europeo presso uno dei Partners Exchange di Ateneo. Si può partecipare anche se si ha già esaurito i mesi Erasmus.

\item Summer School: le Summer Schools offrono la possibilità di andare in un'università, estera o anche italiana, per frequentare corsi estivi di approfondimento su tematiche o settori del proprio corso di laurea, della durata di alcune settimane. I corsi seguiti durante la Summer School possono essere riconosciuti come CFU a scelta, ma solo su richiesta dello studente e senza che debba esserci un accordo tra l'università di appartenenza e quella ospite.\\
Spesso i bandi per le Summer School sono disponibili sui siti delle facoltà che le organizzano, piuttosto che su quello di ateneo.
\end{itemize}


\subsection{Per maggiori informazioni}
Per maggiori informazioni, potete consultare il sito internet della Bicocca all'indirizzo http://www.unimib.it/go/45776/Home/Italiano/Menu-sinistra/Internazionalizzazione/Mobilita-internazionale \\
Per un contatto diretto con l'ufficio della Bicocca che si occupa degli Erasmus, potete scrivere a: international.office@unimib.it
