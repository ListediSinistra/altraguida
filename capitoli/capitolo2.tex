\mysec{L'università}

\subsection{Le Scuole ed i Dipartimenti}

Come previsto dal nuovo Statuto d'Ateneo le unità fondamentali della struttura universitaria sono i Dipartimenti: ogni dipartimento dovrà prendere le decisioni riguardo la didattica e la gestione delle risorse per la ricerca. Tuttavia c'è la possibilità che i Dipartimenti si possano appoggiare ad altre strutture per la gestione della didattica: le Scuole.Le Scuole sono strutture che, nella maggior parte delle università italiane, coordinano, per lo più sotto il profilo amministrativo e didattico, corsi di studio di pari o diverso livello, generalmente afferenti ad aree disciplinari affini. Assomiglia come idea alla vecchia Facoltà, che è spesso confusa con il corso di studi: in realtà gli studenti afferiscono non direttamente alle Scuole, ma ad un Corso di Laurea, il singolo percorso disciplinare che porta al conseguimento di un titolo di studio.
Il Dipartimento si occupa anche dell'organizzazione di uno o più settori di ricerca omogenei per fini e per metodo e dei docenti (professori ordinari, associati e ricercatori) di materie a essi afferenti. Professori che fanno ricerca nei Dipartimenti sono quindi docenti della Scuola e dei Corsi di Laurea.
Facciamo un esempio per schiarirci le idee: Matematica è un corso di studio che afferisce alla Scuola di Scienze; esiste anche il Dipartimento di Matematica che però, a dispetto della Scuola di Scienze, non si occupa esclusivamente della didattica del corso ma anche dell’organizzazione della ricerca.
Per quanto riguarda la didattica nel sistema universitario, si distinguono i seguenti ruoli accademici:
\begin{itemize}
\item professore ordinario (o professore di prima fascia): è il livello più alto della docenza universitaria;
\item professore associato (o professore di seconda fascia): è il secondo livello della docenza a cui si accede superando una selezione effettuata da una commissione nazionale;
\item professore a contratto: è un esterno che viene chiamato dall’università per tenere corsi sulla base di competenze specifiche;
\item ricercatore: figura che si occupa di ricerca e che, per effetto di leggi approvate in passato, deve svolgere ore di didattica integrativa (esercitazioni o laboratori) o tenere un intero corso. Si può diventare ricercatore dopo aver conseguito il Dottorato di Ricerca e aver superato una selezione effettuata da una commissione di Facoltà.
\end{itemize}
Esercitazioni, laboratori e altre attività complementari alle lezioni frontali possono essere tenute anche da un tutor, cioè un dottorando o uno studente iscritto alla Laurea Magistrale.

\subsection{Corsi di Laurea e Laurea Magistrale}
Per la maggior parte dei casi i percorsi di studio prevedono una Laurea Triennale, che può essere seguita da una Laurea Magistrale della durata di due anni. Il Corso di Laurea è il primo tipo di percorso universitario, cui si accede dopo il diploma. Generalmente la durata è di tre anni accademici.
L’obiettivo è quello di “ assicurare allo studente un’adeguata padronanza di metodi e contenuti scientifici generali, nonché l’acquisizione di specifiche conoscenze professionali” (testo della legge 509/99 che ha introdotto il cosiddetto 3+2).
Nella pratica ci si può trovare di fronte a diverse situazioni: in alcuni casi infatti i Corsi di Laurea danno conoscenze complete e contemporaneamente specifiche e ben coordinate; in altri invece la contrazione dei vecchi corsi di 4 o 5 anni ha portato ad un numero troppo elevato di esami con un peso in crediti mal proporzionato, con  conseguente difficoltà a terminare il percorso nei tempi stabiliti. 
La possibilità di trovare una buona occupazione al termine dei primi tre anni dipende dal percorso seguito. In alcuni casi è preferita la giovane età e la possibilità dell'azienda di " formare" il neolaureato; in altri invece è richiesta una maggiore specializzazione e di conseguenza una Laurea Magistrale.
Il Corso di Laurea Magistrale può essere intrapreso dopo la laurea “triennale” ed è rivolto a chi intende proseguire la propria formazione e approfondire quanto studiato nei primi tre anni. La durata prevista è di due anni e lo scopo è “fornire allo studente una formazione di livello avanzato per l’esercizio di attività di elevata qualificazione in ambienti specifici”.
Per accedere al Corso di Laurea Magistrale è necessario soddisfare dei requisiti curricolari esplicitati nel regolamento didattico, che possono consistere nel possesso di una laurea appartenente a determinate classi di laurea oppure nel possedere un certo numero di crediti acquisiti in determinati settori disciplinari.Verificati i requisiti curricolari, è necessario superare una prova  che in genere consiste in un semplice colloquio di ammissione nel quale sarà indicato se è necessario colmare gli eventuali debiti formativi.
Vi sono però delle eccezioni, ovvero corsi di studio che non sono strutturati come 3+2 ma come un solo percorso a ciclo unico. Ne sono esempio Medicina e Chirurgia e Giurisprudenza, strutturati in un unico percorso rispettivamente di sei e cinque anni.Esistono poi Scuole di Specializzazione post lauream, che possono durare dai due ai sei anni e che sono a numero chiuso. Quelle attive nel nostro ateneo sono legate all’area Medica e Chirurgica, Psicologica e dei Servizi Clinici. Ve ne sono altre attivate in collaborazione con differenti Università di Milano.
Dopo la laurea magistrale si può proseguire con un Dottorato di Ricerca a cui si è ammessi superando un concorso pubblico annuale sulla base della tesi di Laurea, di eventuali pubblicazioni  e dell’esito di prove scritte e orali. La durata è non inferiore ai tre anni e in genere non supera i quattro. Durante il Dottorato si svolgono attività di ricerca e si ha la possibilità, o in alcuni casi l’obbligo, di frequentare corsi e seminari.
Esistono poi i Master: corsi altamente professionalizzanti della durata di uno o due anni. Si differenziano in Master di primo livello a cui si può accedere con una laurea triennale e Master di secondo livello per i quali è richiesta una laurea magistrale. In genere sono promossi dall'Università, in collaborazione con strutture di formazione e aziende. Il costo dei Master è solitamente molto elevato (raramente inferiore ai 2000 euro).

\subsection{La gestione dell’Università: la Governance}
Da quest'anno, come già sottolineato in precedenza, grazie al nuovo Statuto la struttura all'interno dell'Ateneo è modificata. All’interno dell’Università esistono diverse istituzioni accademiche:-Rettore: è eletto tra i professori ordinari dell’università ed è la massima autorità accademica. Il mandato dura sei anni e non è rinnovabile.
\begin{itemize}
   \item Consiglio di Amministrazione (CdA): è l'organo che esercita le funzioni ad indirizzo strategico,  amministrativo, finanziario ed economico patrimoniale dell'Università. Ha inoltre il compito di approvare il bilancio. In quest'organo ci sono due rappresentanti degli studenti.
   \item Senato Accademico (SA): è l'organo che definisce la politica generale dell'Università. In particolare si occupa di formulare i piani di sviluppo dell'Università e di promuovere le attività didattiche e scientifiche. Di questo organo fanno parte il Rettore, quattro Direttori di Dipartimento, sei professori, due ricercatori di ruolo e tre rappresentanti degli studenti.
   \item Consiglio degli Studenti (CdS): è composto da 19 studenti che rappresentano tutte le Scuole e due dottorandi; è un organo consultivo composto esclusivamente da studenti ed ha la facoltà di esprimere il parere su temi quali il diritto allo studio, gli importi delle tasse e dei contributi, il regolamento didattico d’ateneo.
   \item Consiglio di Dipartimento e di Scuola: è composto dai professori di prima e di seconda fascia (ordinari, straordinari, associati) afferenti al Dipartimento, dai ricercatori e da un numero variabile di rappresentanti degli studenti. Propone e coordina la attività formative, della didattica e decide sulle questioni attorno alla ricerca. Qualora più Dipartimenti istituissero una Scuola allora viene definita una commissione paritetica nella quale gli studenti sono rappresentati, così come nel Consiglio di Scuola.
   \item Consiglio di Coordinamento Didattico (CCD): si occupa di uno specifico corso di studi nei termini della didattica. È prevista una rappresentanza degli studenti in base al numero dei docenti.
   \item Comitato per lo Sport: organo preposto al coordinamento e alla promozione delle attività sportive per gli studenti e i dipendenti. Sono previsti 2 rappresentanti degli studenti.
\end{itemize}

Un altro organo per il quale gli studenti possono votare è il Consiglio Nazionale degli Studenti Universitari (CNSU): il CNSU è composto da ventotto rappresentanti degli studenti, tra i quali vengono nominati alcuni membri del CUN (Consiglio Universitario Nazionale),organo che ha il compito di prendere decisioni fondamentali sulle riforme in atto e di delineare le linee direttrici dello sviluppo e cambiamento dell'istituzione universitaria nel suo complesso. Il CNSU ha il compito di esprimere pareri sugli atti rilevanti del governo e di porre al Ministro dell'Istruzione quesiti sulla didattica e  sulla condizione studentesca nell'ambito del sistema universitario. Nel corrente mandato l’Università di Milano-Bicocca, nella persona di Sara Capasso, ha la sua rappresentanza sia nel CNSU che nel CUN. 
