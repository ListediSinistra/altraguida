\mysec{Studiare in Bicocca}
La Bicocca offre per l'anno accademico 2013/14 un'ampia scelta di corsi. L'offerta formativa conta infatti 32 corsi di laurea, 34 corsi di laurea magistrale e 4 corsi di laurea magistrale a ciclo unico. I corsi di studio offerti dall'Università sono riconducibili ad otto aree tematiche, sovrapposte alle vecchie facoltà che  con l'introduzione delle legge Gelmini e del nuovo statuto di ateneo cessano di esistere; queste aree sono: Economia, Giurisprudenza, Medicina e Chirurgia, Psicologia, Scienze della formazione, Scienze matematiche fisiche e naturali, Scienze statistiche e Sociologia. 

\subsection{Accesso ai corsi}
Per i corsi di laurea triennale e a ciclo unico della facoltà di Medicina e Chirurgia e per Scienze della Formazione Primaria, il numero di iscritti è stabilito dal MIUR a livello nazionale. Sono previste prove di ammissione con programmi e date di svolgimento uguali in tutta Italia. Per altri corsi di studio il numero di posti è stabilito annualmente da ciascun ateneo.\\
Per i corsi di studio ad accesso libero, invece, è prevista una prova obbligatoria di valutazione alla preparazione iniziale (VPI). La prova ha lo scopo di verificare se la preparazione acquisita durante il percorso scolastico delle scuole superiori sia adeguata ai prerequisiti disciplinari di base fissati dal corso di laurea prescelto. 

\subsection{Lezioni e laboratori}
I corsi di studio sono strutturati in crediti formativi universitari (CFU), l'unità di misura dell'impegno medio di uno studente. Ogni anno di corso comporta l'acquisizione di 60 CFU, ripartiti principalmente in insegnamenti e laboratori. Ogni insegnamento può essere suddiviso in ore di lezione frontale e ore di esercitazione. I laboratori, al contrario, non prevedono lezioni frontali, e di norma è prevista una frequenza obbligatoria al 75%. Gli insegnamenti e i laboratori sono incastrati nel piano di studi. Capire quali corsi sono obbligatori e quali opzionali dovrebbe essere il primo compito della matricola. Inoltre, proseguendo con i semestri, lo studente potrà correggere il piano, avendo acquisito nuove conoscenze e, magari, nuovi interessi. 

subsection{Piani di studi}
Il piano di studi è l'elenco degli esami che lo studente deve sostenere per conseguire la laurea, comprendendo sia quelli obbligatori per ogni corso di laurea sia quelli a scelta dello studente. Il piano di studi viene compilato tramite le Segreterie On Line seguendo la procedura indicata, in un periodo stabilito annualmente dall'Ateneo. Non è necessario compilarlo ogni volta, se non si ha intenzione di modificare quello dell'anno passato. \\
Ogni Corso di Laurea ha un regolamento che stabilisce il numero di esami, la tipologia e il numero di crediti da acquisire; quindi per avere un piano di studi in regola controllate le norme del vostro corso. Per maggiori indicazioni su come si compili un piano di studi, consultate la pagina del sito d'Ateneo o della vostra Facoltà. I piani di studi che non dovessero essere approvati possono essere ripresentati l'anno successivo o modificati quando si consegna la domanda di laurea.

\subsection{Esami} 
Dopo aver frequentato i corsi, arriva il momento di sostenere l'esame e portare a casa un voto, con l'annesso bagaglio di crediti. Le sessioni di esame sono tipicamente tra gennaio e febbraio e tra giugno e settembre, dopo il termine delle lezioni, ma può esserci una grande variabilità a seconda del corso di studio. L'esame può essere scritto o orale. \\
ATTENZIONE: Da regolamento, anche quando l'esame è scritto è previsto un momento di confronto con il docente, in cui potete visionare il compito.\\ 
Alcuni esami sono definiti dalla facoltà propedeutici ad altri. Questo significa che per potersi iscrivere all'esame del corso B, e sostenere l'esame, viene richiesto il superamento dell'esame A. Si presti quindi molta attenzione al regolamento del corso di studi, dove sono indicate tutte le propedeuticità.

\subsection{Inglese e informatica}
Tutti i corsi di studio prevedano prove di conoscenza della lingua inglese (o di un'altra lingua dell'Unione Europea) e di informatica. Non tutti i corsi prevedono entrambe le prove: i corsi dell'area scientifica, per esempio, non richiedono quella di informatica.Le prove di lingua e di informatica sono propedeutiche a tutte le attività del secondo anno. In altre parole, non sarà possibile sostenere esami del secondo anno senza averle prima sostenute. Un buon consiglio è quello di levarsi il pensiero al più presto, approfittando delle conoscenze fresche di scuola superiore. 

\subsection{Trasferimenti}
Le regole specifiche per il trasferimento da un corso di studio, anche di un altro ateneo, a un altro sono determinate dai regolamenti dei singoli corsi di studio. Riguardo al riconoscimento dei crediti già acquisiti, il regolamento studenti garantisce il riconoscimento di almeno il 50% dei CFU per corsi di studio afferenti alla stessa classe (per esempio, tutti i corsi di laurea che si chiamano "fisica"). In ogni caso, è  sempre meglio prendere contatto con il referente del corso di studio di arrivo. 

\subsection{Volete saperne di più?}
Riguardo l'accesso ai corsi e l'offerta formativa: http://www.unimib.it/go/183871592 
Per le prove di lingua e di informatica: http://www.didattica.unimib.it 
Raccolta di informazioni esaustive sui corsi: http://www.unimib.it/go
Ovviamente, potete trovarci e conoscerci nelle aulette rappresentanti al piano -1 dell'U6 (di fianco ai distributori automatici) e al piano 0 dell'U2. 
