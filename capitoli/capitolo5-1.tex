Capitolo 5: Vita all'interno dell'Università
L'ateneo mette a disposizione degli studenti diverse forme di servizi; tutto ciò serve a garantire un'offerta complessiva che punta a standard qualitativi sempre più alti. 

5.1 Diritto allo studio e borse di studio
Trova il suo fondamento nei commi 3 e 4 dell'art.34 della Costituzione, nei quali si afferma il diritto dei capaci e meritevoli, anche se privi di mezzi economici, di raggiungere i gradi più alti degli studi, nonché il dovere della Repubblica a rendere effettivo questo diritto. 
A questo proposito, è stato costituito il Consorzio Interuniversitario per il Diritto allo Studio (CIDiS), al quale partecipano quattro università: Bicocca, Statale, Insubria e IULM.  Esso si occupa di erogare servizi in favore degli studenti meno abbienti, bandendo borse di studio, buoni mensa e alloggi nelle residenze universitarie (http://www.unimib.it/go/page/Italiano/Vivereluniversita/Vivere-lUniversita/Residenze-e-alloggi). Per maggiori informazioni sul Consorzio http://web.consorziocidis.it. 
Oltre ai servizi del CIDiS, accessibili agli studenti delle quattro università consorziate, la Bicocca ha previsto dei servizi di aiuto e sostegno economico per favorire coloro che più ne necessitano, dedicate esclusivamente ai suoi studenti. Per questo motivo esistono due tipi di borse di studio erogate: quelle del CIDiS e quelle d'Ateneo. 

Borse di studio CIDiS
La sovvenzione (accessibile dagli studenti di tutte le università consorziate) consiste, in parte, in un pagamento in denaro e, in parte, in servizi, quali il tesserino mensa o l'agevolazione per l'alloggio. Viene assegnata sulla base di requisiti di reddito e merito ed è erogata in due rate nel corso dell'anno accademico. Per gli studenti del primo anno che la richiedono, viene considerato solo il reddito per l'erogazione della prima rata e, per ottenere anche la seconda, si dovrà conseguire un numero minimo di credito precisato in  ogni bando. Se, negli anni successivi, i mantengono i requisiti previsti, la borsa verrà confermata. Invece, per gli studenti iscritti ad anni successivi al primo, sarà considerato sia il reddito, sia i crediti conseguiti entro il 10 agosto dell'anno accademico precedente. L'ammontare della borsa varia, per quanto riguarda la condizione economica, in base ai tempi di percorrenza del trasporto pubblico necessari per raggiungere la sede del corso di studi. 
I bandi di concorso di quest'anno sono stati pubblicati nel mese di luglio e la domanda può essere presentata fino al 1 ottobre 2012, attraverso lo Sportello On Line accessibile dal sito web del Consorzio. Ulteriori informazioni e dettagli li trovate qui http://web.consorziocidis.it

Borse d'Ateneo
Anche per l'assegnazione di queste sovvenzioni, i parametri valutati sono due: merito (che varia in relazione all'anno di corso del richiedente) e reddito (cioè la condizione economica dello studente, quindi l'ISEEU, dev'essere inferiore a 35.000,00 €). 
In particolare, per gli studenti iscritti al primo anno di corso, il criterio del merito prevede: 
      • di aver ottenuto una valutazione di 90/100 alla maturità; 
      • di aver ottenuto una valutazione di 102/110 al conseguimento della laurea triennale. 
Per gli studenti iscritti al secondo o terzo anno di corso, invece, è previsto: 
      • una media ponderata dei voti di almeno 24/30 
      • di  aver acquisito, al 30 settembre 2011, almeno i  due terzi dei crediti formativi (CFU), arrotondati per eccesso, previsti dal regolamento del corso di studi. 
L'importo di ciascuna borsa è di 4.800,00 € che verrà corrisposto in due rate per gli studenti immatricolati al primo anno e in una sola soluzione per gli iscritti ad anni superiori al primo. I requisiti per accedervi sono sempre merito e reddito.
I termini per la  presentazione delle domande, da effettuare tramite i servizi di Segreteria Online: dal 20 agosto 2012 al 31 ottobre 2012 per gli studenti che si immatricolano al primo anno dei corsi di laurea triennali o laurea specialistica/magistrale a ciclo unico e al corso di laurea in Scienze della Formazione; dal dal 1 al 26 ottobre 2012 per gli studenti che chiedono il rinnovo delle borse di studio riservate a studenti iscritti ad anni successivi al primo; dal 7 gennaio al 31 gennaio 2013 per gli studenti che si immatricolano al primo anno dei corsi di laurea magistrale. 
Per maggiori informazioni: http://www.unimib.it/go/45052
