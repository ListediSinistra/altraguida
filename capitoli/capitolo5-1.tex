\section{Vita all'interno dell'Università}
L'ateneo mette a disposizione degli studenti diverse forme di servizi; tutto ciò serve a garantire un'offerta complessiva che punta a standard qualitativi sempre più alti. 

\subsection{Diritto allo studio e borse di studio}
Trova il suo fondamento nei commi 3 e 4 dell'art.34 della Costituzione, nei quali si afferma il diritto dei capaci e meritevoli, anche se privi di mezzi economici, di raggiungere i gradi più alti degli studi, nonché il dovere della Repubblica a rendere effettivo questo diritto. 
A questo proposito, è stato costituito il Consorzio Interuniversitario per il Diritto allo Studio (CIDiS), al quale partecipano quattro università: Bicocca, Statale, Insubria e IULM.  Esso si occupa di erogare servizi in favore degli studenti meno abbienti, bandendo borse di studio, buoni mensa e alloggi nelle residenze universitarie (http://www.unimib.it/go/9103416085104684826/Home/Italiano/Ateneo/Vivere-lUniversita/Residenze-e-alloggi). Per maggiori informazioni sul Consorzio http://web.consorziocidis.it. 
Oltre ai servizi del CIDiS, accessibili agli studenti delle quattro università consorziate, la Bicocca ha previsto dei servizi di aiuto e sostegno economico per favorire coloro che più ne necessitano, dedicate esclusivamente ai suoi studenti. Per questo motivo esistono due tipi di borse di studio erogate: quelle del CIDiS e quelle d'Ateneo. 

\subsubsection{Borse di studio CIDiS}
La sovvenzione (accessibile dagli studenti di tutte le università consorziate) consiste, in parte, in un pagamento in denaro e, in parte, in servizi, quali il tesserino mensa o l'agevolazione per l'alloggio. Viene assegnata sulla base di requisiti di reddito (Per l'a.a. 2013.14 l'ISEE/ISEU non deve superare € 20.728,45 e l'ISPE/ISPEU non deve essere superiore a € 34.979,27) e merito (che consiste in un determinato numero di crediti da acquisire entro il 10 Agosto dell'anno accademico precedente) ed è erogata in due rate nel corso dell'anno accademico. Agli studenti che si iscrivono al primo anno di un corso di laurea triennale o magistrale a ciclo unico viene richiesto un voto di diploma non inferiore a 70/100, mentre a chi si iscrive ad un primo anno di un corso di laurea magistrale non viene richiesto al momento della presentazione della domanda alcun requisito di merito; in entrambi i casi la prima rata viene pagata in base ai tempi di percorrenza del trasporto pubblico necessari per raggiungere la sede del corso di studi.
Per il pagamento della seconda rata gli studenti iscritti ad un primo anno devono invece conseguire il numero di crediti indicato nel bando entro le scadenze previste. In caso di non raggiungimento di questo requisito la prima rata, già erogata, verrà revocata e dovrà quindi essere restituita. Agli studenti iscritti agli anni successivi al primo viene richiesto, al momento della presentazione della domanda, il conseguimento di un determinato numero di crediti, sulla base dell'anno accademico di prima immatricolazione assoluta.
I bandi di concorso vengono pubblicati alla fine del mese di giugno, la domanda può essere presentata dall'8 luglio fino al 30 settembre 2013, attraverso lo Sportello On Line accessibile dal sito web del Consorzio (www.consorziocidis.it).

\subsubsection{Borse d'Ateneo}
Anche per l'assegnazione di queste sovvenzioni, i parametri valutati sono due: merito (che varia in relazione all'anno di corso del richiedente) e reddito (cioè la condizione economica dello studente, ovvero l'ISEEU, dev'essere inferiore a 20.729,00 € e l'IPSEU inferiore a 34.980,00 €). 
In particolare, per gli studenti iscritti al primo anno di corso, il criterio del merito prevede: 
\begin{itemize}
\item di aver ottenuto una valutazione minima di 90/100 alla maturità; 
\item di aver ottenuto una valutazione minima di 102/110 al conseguimento della laurea triennale. 
Per gli studenti iscritti al secondo o terzo anno di corso, invece, è previsto: 
\item una media ponderata dei voti di almeno 24/30 
\item di  aver acquisito, al 30 settembre 2011, almeno i  due terzi dei crediti formativi (CFU), arrotondati per eccesso, previsti dal regolamento del corso di studi. 
\end{itemize}
L'importo di ciascuna borsa è di 4.800,00 € che verrà corrisposto in una sola soluzione entro 30 giorni dall'assegnazione definitiva delle borse stesse. I requisiti per accedervi sono sempre merito e reddito.
I termini per la  presentazione delle domande, da effettuare tramite i servizi di Segreteria Online: dal 26 agosto 2013 al 31 ottobre 2013 per gli studenti che si immatricolano al primo anno dei corsi di laurea triennali o laurea specialistica/magistrale a ciclo unico; dal dal 1 al 25 ottobre 2013 per gli studenti che chiedono il rinnovo delle borse di studio riservate a studenti iscritti ad anni successivi al primo; dal 7 gennaio al 31 gennaio 2014 per gli studenti che si immatricolano al primo anno dei corsi di laurea magistrale. 
Per maggiori informazioni: http://www.unimib.it/go/45052
