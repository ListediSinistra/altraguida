\mysec{Milan, l`è on gran Milan}
La città di Milano è grande e caotica, affollata da pendolari, turisti, studenti, immigrati, impiegati, curiosi, muratori, operai, ragazzi etc (quasi) tutti rigorosamente... di fretta!Qualsiasi cosa stiate cercando, è molto probabile che la troviate: dalla festa latinoamericana al Cenacolo di Leonardo da Vinci, dai concerti di Vasco agli aperitivi culturali, dagli aperitivi prima di cena ai kebabbari aperti fino a notte fonda.Cercheremo di darvi un'idea molto, molto, generale di tutte le opportunità che vi si presentano.

\subsection{Fare i turisti}
I luoghi di interesse artistico e culturale a Milano sono numerosi.Alcuni sono noti in tutto il mondo, come il Cenacolo, il Duomo, il Castello Sforzesco, la Pinacoteca di Brera, la basilica di Sant'Ambrogio, il teatro alla Scala. Altri sono più nascosti ma altrettanto interessanti come l'Anfiteatro Romano vicino a San Lorenzo alle Colonne, luogo di ritrovo per i giovani; la Chiesa di Santa Maria presso San Satiro, con il finto coro progettato dal Bramante; piazza dei Mercanti, dove parlando in un angolo delle colonne sentirete la vostra voceamplificata dalla parte opposta; l'insolita Torre Velasca...
Il consiglio migliore è quello di prendere la bicicletta (se avete il coraggio di lanciarvi nel traffico cittadino) o il tram e girare senza meta scoprendo la città. Se siete amanti delle due ruote, vi consigliamo il giovedi sera i Critical Mass, giri in bicicletta per Milano con sconti i determinati pub. Quando sarete stanchi, vi consigliamo una sosta in uno dei (purtroppo pochi) parchi rimasti in città, i più famosi sono il Parco Sempione, il Parco di Porta Venezia e parco Forlanini, senza escludere il Parco Nord a pochi minuti dalla nostra università. Per le giornate di pioggia, non mancano i musei e le mostre. Ce ne sono per tutti i gusti: Planetario, Acquario Civico, Museo di Storia Naturale, Museo della Scienza e della Tecnica, la Pinacoteca Ambrosiana, la Triennale (con la sede del design anche in Bovisa) e il Pac, per gli amanti dell'arte moderna. Di recentissima apertura infine il Museo del 900, vicino a Palazzo Reale e un insolito Wow, museo del fumetto aperto negli ultimi mesi.

\subsection{La capitale della moda}
Non ditelo troppo in giro, ma a pochi milanesi capita di fare shopping nel celebre quadrilatero della moda, tanto meno agli studenti, solitamente squattrinati. Prezzi più abbordabili e negozi meno esclusivi si trovano in via Torino, lungo Corso Vittorio Emanuele o in Corso Buenos Aires; ma state attenti che il sabato pomeriggio sono presi d'assalto e vi passerà presto la voglia di spendere i vostri soldi. Se cercate qualcosa di più particolare vi consigliamo la Fiera di Senigallia, appuntamento ogni sabato nella zona di Porta Genova per cercare tra bancarelle di usato e non oggetti fuori dal normale (troverete anfibi, giacche militari, dischi in vinile, fumetti, vestiti colorati, magliette del Che, oggetti da giocoleria).
Quando si avvicinano le feste invece, non potrete perdervi la fiera degli Oh bej! Oh bej! e quella dell'Artigianato, tradizionali appuntamenti durante il ponte di Sant'Ambrogio per milanesi in cerca di regali natalizi.

\subsection{Cinema e teatri}
L'offerta cinematografica in città è ovviamente vasta e varia. Durante il mese di settembre, la stagione viene aperta dal "MilanoFilmFestival" che propone oltre alle proiezioni di corti e lungometraggi, numerosi eventi che animano la città durante tutta la durata del festival. A concluderla invece nel mese di giugno la rassegna "Cannes e dintorni" che propone (spesso in anteprima e in lingua originale) i film che hanno partecipato al celebre Festival francese.Un appuntamento particolare è il "Festival MIX di cinema gaylesbico e queer culture", che propone oltre a corti, lungometraggi, presentazioni di libri, dibattiti e incontri sempre su questa tematica.
In città si trovano numerose sale cinematografiche e multisale, in cui potrete trovare sia le ultime uscite sia iniziative come cineforum, film d'autore, rassegne a tema. Vi ricordiamo che il mercoledì tutti i biglietti hanno un costo ridotto, vi consigliamo anche di controllare particolari offerte (come la promozione "Ricomincio da tre" degli Uci Cinema che offrono ogni martedì un film a 3 euro) o riduzioni per gli studenti.In città sono presenti numerosi teatri, molti offrono riduzioni per studenti o campagne abbonamento che permettono di vedere più spettacoli a prezzi agevolati, vi consigliamo il sito www.lombardiaspettacolo.com per tenerli tutti sotto controllo.

\subsection{Dove si va stasera?}
Un appuntamento classico per gli universitari milanesi è il rito dell'Happy Hour,  servizio offerto ormai da quasi tutti i locali, durante l'orario di cena al prezzo di un cocktail potrete mangiare a volontà da un (solitamente) ricco buffet. I costi variano dai 5 ai 15 euro, quelli con un rapportoqualità/prezzo più conveniente sono, secondo noi: il Maga Furla (vicino all'università e comodo se finite tardi), il Ciu's in via Spontini (frequentato da molti studenti erasmus), il Blender Bar in piazzale Susa (sempre pieno ma con un buffet degno dei migliori pasti luculliani), l'Hora Feliz vicino a San Lorenzo (locale piccolo, ma con numerosi tavoli in strada e cibo ottimo in gran quantità) e, sempre per restare in zona, l'Yguana. Se volete una cena con calma, vi consigliamo le guide prodotte da Terra di Mezzo: appaMilano (che presenta una rassegna tra i più buoni e convenienti ristoranti milanesi) e PappaMondo (guida completa, precisa e pratica di tutti i ristoranti etnici presenti a Milano).
Anche per vedersi dopocena non mancano i locali dove trovarsi. Le zone più frequentate per una passeggiata o un cocktail con gli amici sono quelle di Brera, dei Navigli e delle colonne di San Lorenzo; mentre in Corso Como e vicino al Parco Sempione troverete i locali più costosi e le discoteche più selettive. Si possono trovare posti interessanti anche in zone più lontane dal centro, come il Frida (via Pollaiuolo 3), una vecchia fabbrica ristrutturata in giardino, enoteca, art gallery, cocktail bar e molto altro; oppure il Turnè (via Paolo Frisi 3) piccolo locale in zona Porta Venezia ma con una programmazione diversa e varia ogni sera della settimana (dallo spritz a 3 euro il martedì al cinema domenicale, passando per degustazioni di vini e offerte speciali per i cocktail). In estate non si può mancare all'appuntamento con il Carroponte a sesto San Giovanni che da qualche anno offre concerti a prezzo popolare all'aperto. 
Se siete amanti della birra, non potete perdervi il Birrificio Lambrate (via Adelchi 3) che offre una selezione di birre di propria produzione; l'appuntamento il giovedì sera è in zona centrale per l'offerta di due medie a 5 euro dell'Outback (via Carlo Tenca 10) mentre la domenica e il lunedì l'EastEnd vicino al Cimitero di Lambrate vi offre una pinta al costo della piccola. Per essere sicuri di non perdervi nessun appuntamento e rimanere aggiornati sulle nuove aperture e sulle offerte speciali (tra cui molte rivolte alla clientela universitaria) vi consigliamo la guida Zero (distribuita gratuitamente ogni quindici giorni in locali e negozi) e gli inserti settimanali di alcuni giornali, come ViviMilano del Corriere o TuttoMilano della Repubblica. 

\subsection{Centri sociali e circoli Arci}
Il più famoso centro sociale milanese è il Leoncavallo, in via Watteau, offre durante tutto l'anno concerti, incontri, proiezioni cinematografiche, feste, corsi di lingua, laboratori (per maggiori informazioni www.leoncavallo.org). Vi consigliamo vivamente di partecipare a la Terra trema, a novembre con degustazioni di vini da tutta Italia, e la festa del raccolto e della semina..Leggendo i manifesti sui muri della città o i giornali, scoprirete che in quasi ogni quartiere si trova un vecchio edificio abbandonato, ora occupato e autogestito da collettivi che propongono manifestazioni ed eventi; ad esempio il Casa Loca, vicino alla Bicocca, dove vi offriranno un piatto di pasta a 3 euro e partite al calcetto infinite, la Cascina Torchiera in zona Certosa o il centro sociale ZAM in corso di Porta Ticinese. Nella medesima zona, in uno degli ex caselli daziari di Piazza XXIV Maggio, c'è il Lato B, uno spazio sociale gestito da un'associazione di studenti e giovani lavoratori dove si può fare una partita a calcetto, organizzare una festa, suanare e anche studiare negli orari di aula studio. Per entrare nello spazio serve la tessera dell'associazione acquistabile a 5 euro.
I circoli Arci presenti tra città e provincia sono più di un centinaio, si tratta di Associazioni con uno luogo di ritrovo dove solitamente si trovano cibo e bevande a prezzi molto convenienti, si organizzano concerti (spesso di artisti emergenti o "non convenzionali"), è possibile partecipare a dibattiti, cineforum e molto altro. 
Per entrare è necessario avere la tessera (il cui costo è attorno ai 10 euro) che ha validità annuale e permette l'ingresso in tutti i circoli Arci d'Italia.Potete consultare il sito www.arcimilano.it per avere l'elenco completo, vi segnaliamo il circolo Magnolia (via Circonvallazione-Segrate) dove si organizzano concerti e serate che attirano giovani da tutta la provincia, il circolo Metissage (quartiere Isola, dietro la stazione di Porta Garibaldi) se volete godervi una serata tranquilla in compagnia della buona musica e l'Agorà, circolo dell'hinterland milanese che offre spettacoli alternativi e di gruppi emergenti.  

\subsection{Trasporto pubblico}
Il trasporto pubblico milanese non è all'altezza di quello delle altre grandi città europee, ma si difende con dignità ed è tutt'ora oggetto di continue modifiche e miglioramenti (speriamo). L'A.T.M. gestisce quattro linee di metrò che collegano le periferie con il centro della città, numerosi autobus e tram che coprono in modo più capillare il territorio urbano ed extraurbano. Inoltre il cosiddetto "passante", ovvero le linee suburbane di Trenord, agevola gli spostamenti verso comuni leggermente più distanti. 
Il biglietto urbano costa 1,50 euro (valido per un'ora e un quarto per spostamenti interni alla città e una sola corsa in metro), ma se siete assidui frequentatori dei mezzi pubblici potrebbe risultarvi più conveniente l'abbonamento mensile (17 euro) o quello annuale (170 euro) riservato agli studenti fino ai 26 anni. Per maggior comodità, potrete acquistare un biglietto giornaliero (4,50 euro per 24 ore) o un carnet da dieci viaggi (utilizzabili in momenti diversi) o potreste provare la nuova tessera RicaricaMi su cui caricare biglietti, settimanali o carnet da avere sempre a portata di mano nel portafoglio.
I mezzi pubblici sono comodi e veloci durante il giorno per evitare il traffico ed il parcheggio a pagamento. In settimana l'ultima corsa della metropolitana è verso mezzanotte, mentre da quasi due anni il servizio notturno per il week end è stato finalmente potenziato con autobus sostitutivi delle linee metropolitane che girano tutta la notte e nuove linee notturne; un servizio alternativo è però quello del Radiobus, un bus prenotabile per spostarsi da dove a dove volete fino alle 2 di notte.
Per maggiori informazioni sui servizi, sul costo dei biglietti, per poter calcolare il percorso più breve e quant'altro potete consultare il sito www.atm-mi.it o chiamare il numero verde 800 80 81 81. 
