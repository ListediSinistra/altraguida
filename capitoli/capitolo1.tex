\mysec{L'AltraGuida}
\subsection{Cos’è l’AltraGuida}
Questa guida è stata scritta e progettata soprattutto per coloro che stanno per iniziare la propria esperienza universitaria, le nuove matricole. L'approccio con il mondo universitario non è certo privo di ostacoli: nuovi spazi, procedure burocratiche complesse e macchinose... E per uno studente fuori sede la situazione è ancora più complicata: bisogna cercare una casa, ambientarsi in una nuova città. Una guida scritta da studenti per studenti può quindi fornire qualche consiglio e risolvere parte dei dubbi, cercando di chiarire il "mondo accademico", nonostante alcuni aspetti del sistema universitario siano complessi, rigidi e talvolta mal funzionanti.
Ulteriore speranza è che questa guida possa tornare utile anche a chi in Bicocca ci è arrivato qualche anno fa. Perché non si finisce mai di imparare e perché, a volte, nella fretta di terminare gli studi nel minor tempo possibile, si rischia di convincersi che l'Università non sia nient'altro che un insieme di lezioni, laboratori, tirocini ed esami. In Bicocca, e più in generale a Milano, ci sono anche molte altre opportunità e attività (sociali, culturali, politiche, sportive. . . ) che non si trovano sulla Guida dello Studente, ma che questa guida "alternativa" prende in considerazione.

\subsection{Chi siamo e cosa abbiamo fatto: LdS}
ListediSinistra è un associazione nata nel 2001 da studenti universitari che, pur non riconoscendosi necessariamente in organizzazioni politiche, sentivano l'esigenza di prendere parte alla vita universitaria esprimendo i valori di democrazia, solidarietà sociale, libertà e laicità dell'istruzione e diritto allo studio.
La partecipazione alla vita politica dell'Ateneo si svolge tramite un'estesa rete di Rappresentanti di diversi Corsi di Laurea eletti in quasi tutti gli organi accademici dell'Università Milano-Bicocca, ove ListediSinistra risulta essere la prima lista.
I membri dell'associazione credono infatti nella possibilità dell'intervento degli studenti all'interno dell'università; consapevoli di non poter modificare ogni elemento negativo ed ogni malfunzionamento presente, ritengono tuttavia che sia comunque utile essere sempre pronti a richiamare l'attenzione dell'istituzione sulle problematiche più significative e, in particolare, ad ascoltare e aiutare gli studenti lungo il loro percorso accademico.
Inoltre i rappresentanti di ogni corso di laurea si impegnano a ricevere gli studenti per informazioni o chiarimenti, risultando di fatto un sicuro punto di riferimento per qualsiasi problema, e a fornire alcuni utili "servizi" con la collaborazione e il supporto dell'Università degli studi Milano-Bicocca.
Ne sono un esempio la stesura di questa guida o la creazione di un servizio Bacheca Alloggi per favorire il mercato degli affitti per gli studenti fuorisede.
ListediSinistra ha infatti organizzato tre apposite bacheche in cui concentrare tutti gli annunci di ricerca e di offerta alloggi collocate una nell'edificio U6 piano -1, lato mensa, una nell'edificio U3 piano -1 (vicino all'aula U3/1 e al bar) ed una in U2, piano terra (di fronte all’aula rappresentanti).
Ma ListediSinistra organizza anche incontri, eventi culturali, spettacoli teatrali, dibattiti e conferenze sia su temi universitari che di attualità quali ad esempio la Resistenza, la memoria storica, i diritti sociali, la situazione internazionale.In questi anni sono stati ottenuti numerosi importanti risultati tra i quali l'aumento del numero di borse studio e la riforma della tassazione a beneficio dei redditi medio-bassi. Per l'anno accademico che sta per cominciare siamo riusciti a far sì che la prima rata della tassa universitaria fosse ridotta di 40€ a fronte di una maggiore trasparenza circa il reinvestimento della tassa stessa da parte dell'istituzione; abbiamo contribuito a fa sì che una gestione oculata delle risorse negli anni permettesse ulteriori investimenti per le borse agli studenti che intendono affrontare un programma Erasmus. Durante questo mandato ci siamo trovati a confrontarci con una riforma del sistema universitario di grossa portata. Ci siamo impegnati affinché il ruolo degli studenti nella futura struttura sia partecipativo ai grandi momenti decisionali che fanno parte della vita dell'ateneo e che la influenzano quotidianamente.

\subsection{Chi saremo e cosa faremo: le nuove elezioni studentesche}
A seguito delle riforma Gelmini l'università si è dovuta preparare ad una inevitabile stagione di cambiamenti; nell'ultimo anno come rappresentanti degli studenti siamo stati protagonisti nella ridefinizione dello Statuto d'Ateneo che definisce le direttive e la forma organizzativa della nostra università. Ora siamo intenzionati a far sì che l'università nella quale viviamo tutti i giorni possa affrontare al meglio le nuove sfide per migliorarsi e migliorare la vita degli studenti al suo interno.
Ed è proprio a questo fine che stiamo cercando di attingere al maggior numero di spunti differenti; in quest'autunno si svolgeranno le nuove elezioni studentesche e riteniamo fondamentale che lo studente ne sia a conoscenza in quanto grande opportunità di crescita e confronto.
Per proseguire su questa strada è quindi necessario che ogni persona interessata a contribuire ad un miglior rapporto con l'istituzione possa trovare un luogo nel quale esprimersi e confrontarsi con altre “teste”: questo è quello che ListediSinistra vuole essere nel futuro come lo è stata nel passato. Se sei interessato non esitare a cercarci ed incontrarci presso le aule rappresentanti, in U2 al piano terra ed in U6 al piano -1.

