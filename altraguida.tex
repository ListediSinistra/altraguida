\documentclass[12pt,a4paper,twoside]{article}
\usepackage[T1]{fontenc}
\usepackage[utf8x]{inputenc}
\usepackage[italian]{babel}
% \usepackage[utf8x]{inputenc}
\usepackage{amsmath,amssymb,amsfonts,textcomp}
\usepackage{color}
\usepackage{calc}
\usepackage{eurosym}
\usepackage[dvips]{geometry}
\usepackage{eso-pic}
\usepackage{graphicx}
%\usepackage{makeidx}
\usepackage{fancyhdr,fancybox}
%\usepackage[none,light,timestamp]{draftcopy}
\usepackage{multicol}
\usepackage{enumitem}
\usepackage{url}
\usepackage[breaklinks=true]{hyperref}
%%%%%%%%%%
%%%%%%%%%% 
\usepackage{sectsty}
\usepackage{ifthen}
\usepackage{sectionbox}
% \usepackage{skull}
\usepackage{lettrine}

\doublesectionbox
\shadowsubsectionbox
\shadowsubsubsectionbox


\definecolor{sectboxrulecol}{rgb}{0,0,0}
\definecolor{sectboxfillcol}{rgb}{0,0,0}
\definecolor{sectboxtextcol}{rgb}{1,1,1}

\definecolor{subsectboxrulecol}{rgb}{0.75,0.75,0.75}
\definecolor{subsectboxfillcol}{rgb}{0.9,0.9,0.9}
\definecolor{subsectboxtextcol}{rgb}{0,0,0}

\definecolor{subsubsectboxrulecol}{rgb}{0.75,0.75,0.75}
\definecolor{subsubsectboxfillcol}{rgb}{0.9,0.9,0.9}
\definecolor{subsubsectboxtextcol}{rgb}{0,0,0}

\newcommand{\mysecname}{INDICE}
\newcommand{\mysec}[1]{%\null\vskip24pt%
\gdef\mysecname{\uppercase{#1}}
%\renewcommand{\mysecname}{\uppercase{#1}}
\begin{sectionbox}{#1}\relax
\end{sectionbox}}
\sectionfont{\sffamily}%\textcolor[gray]{0.2}}

\newcommand{\mysubsecname}{\relax}
\newcommand{\mysubsec}[1]{%\null\vskip12pt%
%\renewcommand{\mysubsecname}{#1}
\gdef\mysubsecname{#1}
\relax\vskip6pt \begin{subsectionbox}{#1}\relax
\end{subsectionbox}}
\subsectionfont{\sffamily}%\textcolor[gray]{0.2}}

\newcommand{\mysubsubsec}[1]{%\null\vskip6pt%
\relax\vskip6pt \begin{subsubsectionbox}{#1}\relax
\end{subsubsectionbox}}
\subsubsectionfont{\sffamily}%\textcolor[gray]{0.2}}

\makeatletter\renewcommand\paragraph{\@startsection{paragraph}{4}{\z@}%
                                    {12pt \@plus1ex \@minus.2ex}%
                                    {-1em}%
                                    {\normalsize\bfseries\sffamily}}
\makeatother         

%\paragraphfont{\sffamily}
                           
%%%%%%%%%%
%%%%%%%%%% 


\geometry{top=0.6in,bottom=0.7in,left=0.7in,right=0.7in}
%\geometry[E]{top=0.7in,bottom=0.7in,left=0.7in,right=0.6in}
%\hypersetup{colorlinks=true, linkcolor=blue, filecolor=blue, pagecolor=blue, urlcolor=blue}
%\draftcopySetGrey{0.90}
%\draftcopyName{Bozza}{150}


%\newcommand{\collegamento}[2]{\catcode`\%11\relax {\href{#1}{\underline{#2}}}\catcode`\%14\relax}

%\newcommand{\collegamento}[2]{\href{#1}{\underline{#2}}}

\newcommand{\collegamento}[2]{\href{#1}{\tt #2}}

\newcommand{\collegamail}[1]{\collegamento{mailto:#1}{#1}}

\newcommand{\gdstud}{ \collegamento{http://www.unimib.it/upload/bicocca\%20guida\%20\%202007\%20-\%20INTERNET\%20-.pdf}{Guida dello
Studente}}

%\newcommand{\LdS}{{\textbf{Listedi}}{\textbf{\textcolor[rgb]{0.8,0.0,0.2}{Sinistra}}}}
%\newcommand{\LdS}{Listedi\textcolor[rgb]{0.8,0.0,0.2}{Sinistra}}
\newcommand{\LdS}{{\textbf{Listedi}}{\textbf{\textcolor[rgb]{0.3,0.3,0.3}{Sinistra}}}}

\newcommand{\tab}{\hspace{5mm}}

\newcommand\mynote[1]{\index{aaa@\textbf{Tom:}}\marginpar%
{\raggedright\rule{.08\marginparwidth}{12pt}\vbox{\raisebox{-11pt}{\hbox{\hskip 2pt \scriptsize{\sl Tom:}}}\\\rule{.92\marginparwidth}{.4pt}}\\ \footnotesize{#1}}
}%

\newcommand{\postit}[1]{\textcolor[rgb]{0,0.8,0.2}{#1}}

\newcommand\NdP[1]{\index{aaa@\textbf{le note di Pietro sono alle pagg.:;}}\marginpar%
{\raggedright\rule{.08\marginparwidth}{12pt}\vbox{\raisebox{-11pt}{\hbox{\hskip 2pt \scriptsize{\sl NdP:}}}\\\rule{.92\marginparwidth}{.4pt}}\\ \footnotesize{#1}}}%
\renewcommand\NdP[1]{\relax}

\newcommand{\boldsansserif}[1]{\textsf{\textbf{#1}}}

%\newcounter\warningcounter
%\setcounter{\warningcounter}{1}
%\newcommand{\warning}[2]{\setcounter\warningcounter#2\fbox{{\ifnum\warningcounter=3 \skull\skull\skull \elseif \ifnum\warningcounter=1 \skull\skull \fi\else \skull \fi} \textbf{ATTENZIONE:}#1}}

%\newcommand{\mywarning}[1]{\noindent\ovalbox{\hskip12pt\begin{minipage}{0.85\columnwidth}$\skull\skull\skull$ \textbf{ATTENZIONE!}\\#1\end{minipage}\hskip12pt}}

\newcommand{\mywarning}[1]{\noindent\ovalbox{\hskip12pt\begin{minipage}{0.85\columnwidth}\lettrine[image=true, lines=2, loversize=0.25]{warn.pdf}{} \textbf{ATTENZIONE!}\\#1\end{minipage}\hskip12pt}}


\newcommand{\vedi}[1]{(vedi p.~\pageref{#1})}

\setlist{noitemsep, topsep=0pt, leftmargin=15pt, labelindent=0pt}
\setlength{\headheight}{15pt}
\setlength{\columnsep}{24pt} 
\setlength{\columnseprule}{0.4pt}   
\setlength{\columnwidth}{10cm}


\pagestyle{fancy}
%\newcommand{\chaptermark}[1]{\markboth{#1}{}}
%\renewcommand{\sectionmark}[1]{\markright{\thesection\ #1}}
\fancyhf{}
\fancyhead[LO]{\sf\bfseries\thepage \quad www.\LdS.org}
\fancyhead[RE]{\LdS @gmail.com \quad \sf\bfseries\thepage}
%\fancyhead[LO]{\sf\bfseries\rightmark}
\fancyhead[LE]{\sf\bfseries \relax\quad\thesubsection{} \quad \mysubsecname}
%\fancyhead[RE]{\sf\bfseries\leftmark}
\fancyhead[RO]{\sf\bfseries \ifnum\thesection=0 \relax \else \thesection \fi{}\quad \mysecname\quad}
\marginparwidth45pt
%\fancyfoot[C]{\includegraphics[height=9mm, keepaspectratio]{logo.pdf}}

\makeindex

%qui si definisce lo sfondo da richiamre con un \AddToShipoutPicture{\Sfonda}
%si disattiva con \ClearShipoutPicture - si veda pacchetto eso-pic -
%\include{sfonda}


\hyphenation{e-ster-no in-di-spen-sa-bi-le tra-sfe-ri-men-ti spe-cia-li-sti-ca ma-tu-ra-ti de-sti-na-zio-ne bru-sco cre-di-ti o-spe-da-li pos-si-bi-li-t\'a i-ni-zio neu-ro-psi-co-lo-gi-a i-nol-tre acqui-sto Se-sto in-va-li-di-t\`a auto-va-lu-ta-zio-ne a-spet-ti sod-di-sfi o-gni Ni-guar-da E-di-fi-cio li-vel-lo luo-ghi o-biet-ti-vi fun-zio-na-men-to con-si-de-ra-zio-ne li-ber-t\'a ap-pro-fit-ta mi-ni-ste-ro in-te-res-se cul-tu-ra-le per-met-ten-do e-ste-ro i-sti-tu-zio-ne ri-chie-sta bre-scia ber-ga-mo e-ma-na-ti li-bri bi-blio-te-che i-nol-tre que-sta in-du-stri-e e-du-ca-ti-ve du-re-ran-no ri-ser-va-te gra-dua-to-ria me-ri-to e-so-ne-ri so-cie-t\'a i-scri-zio-ne bi-blio-te-ca o-rien-ta-men-to or-ga-niz-za-ti-vo re-gi-stra-ti po-ter re-qui-si-ti in-se-ri-ti ri-ce-ve-ran-no que-stio-na-rio in-di-vi-dua-re di-sa-bi-li i-ni-zia-ti-ve a-scol-ta-re pro-ble-ma-ti-che cam-bia-men-to am-mi-ni-stra-zio-ni per-so-na-le scri-vi-ci per-met-to-no si-tu-a-to fa-ci-li-ta-re rap-pre-sen-tan-ti bio-lo-gi-che cur-ri-cu-la ma-gi-stra-le o-biet-ti-vo ca-rat-te-riz-za-zio-ne geo-lo-gi-che af-fac-ciar-si ca-len-da-rio scien-ti-fi-co ma-na-ge-ment sta-ti-sti-ca col-la-bo-ra-zio-ne bi-so-gno-si co-sti-tu-i-sco-no i-ni-zia-ti-va me-dian-te si-tua-zio-ni tes-se-rar-si ri-sto-ran-te qua-li-t\'a so-li-da-le se-ra-te na-vi-gli-o e-stre-ma-men-te in-te-res-san-tis-si-me o-ri-gi-na-le ras-se-gne ca-po-la-vo-ri spin-go-no e-si-to mec-ca-ni-smi pre-oc-cu-pa-te-vi  di-scu-to-no e-di-fi-cio ma-te-ma-ti-ca o-biet-ti-vo la-bo-ra-to-ri si-gni-fi-ca-ti-va me-di-ci-na a-na-li-si si-gni-fi-ca e-sem-pi-o chi-a-ma-to que-sti-o-ni glo-ba-le a-zi-en-da-li-sti-co in-ter-na-zio-na-li ge-ne-ra-li-sta at-ti-vi-t\'a e-du-ca-zio-ne la-bo-ra-to-rio ca-rat-te-riz-za rap-pre-sen-tan-ti fun-zio-na-li cia-scun co-min-cia-no}

\begin{document}
%\AddToShipoutPicture{\Sfonda}
\sf  %font computer modern sans serif 
%
\pagenumbering{roman}
\title{L'\emph{altra}\/guida\footnote{{\sf versione 2.2 del \today}}}
\author{realizzata da \LdS
\footnote{\textsf{grazie al contributo derivante dal bando per le attivit\`a delle associazioni
studentesche dell'Universit\`a degli Studi di Milano--Bicocca \collegamento{http://www.unimib.it}{www.unimib.it}.\newline \null \newline}.}\\ 
Rappresentanti degli Studenti \\ \collegamento{http://www.listedisinistra.org}{www.\LdS.org}\\ 
\collegamento{mailto:listedisinistra@gmail.com}{\LdS@gmail.com}}
\date{Anno Accademico PROVA}

\maketitle

\thispagestyle{empty}

\vskip.5in
\begin{multicols}{2}
\tableofcontents
\end{multicols}
\vskip6pt \hrule \vskip6pt

\begin{multicols}{2}
\pagenumbering{arabic}
 ciaociao

\mysec{L'università}

\subsection{Le Scuole ed i Dipartimenti}

Come previsto dal nuovo Statuto d'Ateneo le unità fondamentali della struttura universitaria sono i Dipartimenti: ogni dipartimento dovrà prendere le decisioni riguardo la didattica e la gestione delle risorse per la ricerca. Tuttavia c'è la possibilità che i Dipartimenti si possano appoggiare ad altre strutture per la gestione della didattica: le Scuole.Le Scuole sono strutture che, nella maggior parte delle università italiane, coordinano, per lo più sotto il profilo amministrativo e didattico, corsi di studio di pari o diverso livello, generalmente afferenti ad aree disciplinari affini. Assomiglia come idea alla vecchia Facoltà, che è spesso confusa con il corso di studi: in realtà gli studenti afferiscono non direttamente alle Scuole, ma ad un Corso di Laurea, il singolo percorso disciplinare che porta al conseguimento di un titolo di studio.
Il Dipartimento si occupa anche dell'organizzazione di uno o più settori di ricerca omogenei per fini e per metodo e dei docenti (professori ordinari, associati e ricercatori) di materie a essi afferenti. Professori che fanno ricerca nei Dipartimenti sono quindi docenti della Scuola e dei Corsi di Laurea.
Facciamo un esempio per schiarirci le idee: Matematica è un corso di studio che afferisce alla Scuola di Scienze; esiste anche il Dipartimento di Matematica che però, a dispetto della Scuola di Scienze, non si occupa esclusivamente della didattica del corso ma anche dell’organizzazione della ricerca.
Per quanto riguarda la didattica nel sistema universitario, si distinguono i seguenti ruoli accademici:
\begin{itemize}
\item professore ordinario (o professore di prima fascia): è il livello più alto della docenza universitaria;
\item professore associato (o professore di seconda fascia): è il secondo livello della docenza a cui si accede superando una selezione effettuata da una commissione nazionale;
\item professore a contratto: è un esterno che viene chiamato dall’università per tenere corsi sulla base di competenze specifiche;
\item ricercatore: figura che si occupa di ricerca e che, per effetto di leggi approvate in passato, deve svolgere ore di didattica integrativa (esercitazioni o laboratori) o tenere un intero corso. Si può diventare ricercatore dopo aver conseguito il Dottorato di Ricerca e aver superato una selezione effettuata da una commissione di Facoltà.
\end{itemize}
Esercitazioni, laboratori e altre attività complementari alle lezioni frontali possono essere tenute anche da un tutor, cioè un dottorando o uno studente iscritto alla Laurea Magistrale.

\subsection{Corsi di Laurea e Laurea Magistrale}
Per la maggior parte dei casi i percorsi di studio prevedono una Laurea Triennale, che può essere seguita da una Laurea Magistrale della durata di due anni. Il Corso di Laurea è il primo tipo di percorso universitario, cui si accede dopo il diploma. Generalmente la durata è di tre anni accademici.
L’obiettivo è quello di “ assicurare allo studente un’adeguata padronanza di metodi e contenuti scientifici generali, nonché l’acquisizione di specifiche conoscenze professionali” (testo della legge 509/99 che ha introdotto il cosiddetto 3+2).
Nella pratica ci si può trovare di fronte a diverse situazioni: in alcuni casi infatti i Corsi di Laurea danno conoscenze complete e contemporaneamente specifiche e ben coordinate; in altri invece la contrazione dei vecchi corsi di 4 o 5 anni ha portato ad un numero troppo elevato di esami con un peso in crediti mal proporzionato, con  conseguente difficoltà a terminare il percorso nei tempi stabiliti. 
La possibilità di trovare una buona occupazione al termine dei primi tre anni dipende dal percorso seguito. In alcuni casi è preferita la giovane età e la possibilità dell'azienda di " formare" il neolaureato; in altri invece è richiesta una maggiore specializzazione e di conseguenza una Laurea Magistrale.
Il Corso di Laurea Magistrale può essere intrapreso dopo la laurea “triennale” ed è rivolto a chi intende proseguire la propria formazione e approfondire quanto studiato nei primi tre anni. La durata prevista è di due anni e lo scopo è “fornire allo studente una formazione di livello avanzato per l’esercizio di attività di elevata qualificazione in ambienti specifici”.
Per accedere al Corso di Laurea Magistrale è necessario soddisfare dei requisiti curricolari esplicitati nel regolamento didattico, che possono consistere nel possesso di una laurea appartenente a determinate classi di laurea oppure nel possedere un certo numero di crediti acquisiti in determinati settori disciplinari.Verificati i requisiti curricolari, è necessario superare una prova  che in genere consiste in un semplice colloquio di ammissione nel quale sarà indicato se è necessario colmare gli eventuali debiti formativi.
Vi sono però delle eccezioni, ovvero corsi di studio che non sono strutturati come 3+2 ma come un solo percorso a ciclo unico. Ne sono esempio Medicina e Chirurgia e Giurisprudenza, strutturati in un unico percorso rispettivamente di sei e cinque anni.Esistono poi Scuole di Specializzazione post lauream, che possono durare dai due ai sei anni e che sono a numero chiuso. Quelle attive nel nostro ateneo sono legate all’area Medica e Chirurgica, Psicologica e dei Servizi Clinici. Ve ne sono altre attivate in collaborazione con differenti Università di Milano.
Dopo la laurea magistrale si può proseguire con un Dottorato di Ricerca a cui si è ammessi superando un concorso pubblico annuale sulla base della tesi di Laurea, di eventuali pubblicazioni  e dell’esito di prove scritte e orali. La durata è non inferiore ai tre anni e in genere non supera i quattro. Durante il Dottorato si svolgono attività di ricerca e si ha la possibilità, o in alcuni casi l’obbligo, di frequentare corsi e seminari.
Esistono poi i Master: corsi altamente professionalizzanti della durata di uno o due anni. Si differenziano in Master di primo livello a cui si può accedere con una laurea triennale e Master di secondo livello per i quali è richiesta una laurea magistrale. In genere sono promossi dall'Università, in collaborazione con strutture di formazione e aziende. Il costo dei Master è solitamente molto elevato (raramente inferiore ai 2000 euro).

\subsection{La gestione dell’Università: la Governance}
Da quest'anno, come già sottolineato in precedenza, grazie al nuovo Statuto la struttura all'interno dell'Ateneo è modificata. All’interno dell’Università esistono diverse istituzioni accademiche:-Rettore: è eletto tra i professori ordinari dell’università ed è la massima autorità accademica. Il mandato dura sei anni e non è rinnovabile.
\begin{itemize}
   \item Consiglio di Amministrazione (CdA): è l'organo che esercita le funzioni ad indirizzo strategico,  amministrativo, finanziario ed economico patrimoniale dell'Università. Ha inoltre il compito di approvare il bilancio. In quest'organo ci sono due rappresentanti degli studenti.
   \item Senato Accademico (SA): è l'organo che definisce la politica generale dell'Università. In particolare si occupa di formulare i piani di sviluppo dell'Università e di promuovere le attività didattiche e scientifiche. Di questo organo fanno parte il Rettore, quattro Direttori di Dipartimento, sei professori, due ricercatori di ruolo e tre rappresentanti degli studenti.
   \item Consiglio degli Studenti (CdS): è composto da 19 studenti che rappresentano tutte le Scuole e due dottorandi; è un organo consultivo composto esclusivamente da studenti ed ha la facoltà di esprimere il parere su temi quali il diritto allo studio, gli importi delle tasse e dei contributi, il regolamento didattico d’ateneo.
   \item Consiglio di Dipartimento e di Scuola: è composto dai professori di prima e di seconda fascia (ordinari, straordinari, associati) afferenti al Dipartimento, dai ricercatori e da un numero variabile di rappresentanti degli studenti. Propone e coordina la attività formative, della didattica e decide sulle questioni attorno alla ricerca. Qualora più Dipartimenti istituissero una Scuola allora viene definita una commissione paritetica nella quale gli studenti sono rappresentati, così come nel Consiglio di Scuola.
   \item Consiglio di Coordinamento Didattico (CCD): si occupa di uno specifico corso di studi nei termini della didattica. È prevista una rappresentanza degli studenti in base al numero dei docenti.
   \item Comitato per lo Sport: organo preposto al coordinamento e alla promozione delle attività sportive per gli studenti e i dipendenti. Sono previsti 2 rappresentanti degli studenti.
\end{itemize}

Un altro organo per il quale gli studenti possono votare è il Consiglio Nazionale degli Studenti Universitari (CNSU): il CNSU è composto da ventotto rappresentanti degli studenti, tra i quali vengono nominati alcuni membri del CUN (Consiglio Universitario Nazionale),organo che ha il compito di prendere decisioni fondamentali sulle riforme in atto e di delineare le linee direttrici dello sviluppo e cambiamento dell'istituzione universitaria nel suo complesso. Il CNSU ha il compito di esprimere pareri sugli atti rilevanti del governo e di porre al Ministro dell'Istruzione quesiti sulla didattica e  sulla condizione studentesca nell'ambito del sistema universitario. Nel corrente mandato l’Università di Milano-Bicocca, nella persona di Sara Capasso, ha la sua rappresentanza sia nel CNSU che nel CUN. 

\mysec{Le segreterie}
In università ci sono due segreterie principali, a cui noi studenti dobbiamo rivolgerci:
\begin{itemize}
\item la segreteria studenti è l'ufficio responsabile dello status dello studente rispetto all'università.Questa segreteria si occupa di pratiche come le immatricolazioni, la consegna delle domande di laurea, il trasferimento da o verso un'altra università o un altro corso di laurea, l'interruzione degli studi...
\item la segreteria didattica è invece l'ufficio responsabile dell'organizzazione della didattica per ogni corso di laurea. Probabilmente avrete molto a che fare con questa segreteria, che si occupa di questioni più quotidiane nella vita dello studente come gestire gli aspetti burocratici degli esami, stabilire l'orario delle lezioni, compilare i piani di studi, gestire la burocrazia dei tirocini e delle tesi di laurea.
\end{itemize}

\subsection{Segreterie On Line}
Tutte le pratiche burocratiche vengono gestite tramite ESSE3, il sistema telematico che l'Università utilizza per gestire le carriere accademiche di tutti gli studenti. 
Ogni studente ha un profilo personale a cui può accedere da qualsiasi computer tramite internet o dai terminali blu che si trovano in giro per l'università. Lo scopo è quello di velocizzare molti aspetti burocratici; questo sistema permette infatti di stampare documenti (come autocertificazioni riguardo la propria carriera accademica o i bollettini MAV per i pagamenti delle tasse), di compilare il proprio piano di studi, di partecipare ai bandi indetti dall'Università, di fare domanda di laurea. L'utilizzo principale delle segreterie online è per l'iscrizione agli esami e per visualizzare il proprio libretto con voti, medie, crediti acquisiti. 
Il sistema è migliorato negli ultimi anni, ma ancora migliorabile; state attenti ed evitate di ridurvi all'ultimo minuto, se ci fossero dei problemi e non possiate completare le vostre pratiche, la segreteria didattica dovrebbe potervi aiutare.
Grazie al nuovo metodo di verbalizzazione online, dopo aver effettuato l'esame, appena il professore chiude il registro i voti vengono caricati automaticamente sulla vostra pagina personale. In questo modo le carriere degli studenti sono aggiornate immediatamente con voti e crediti acquisiti.

\subsection{Immatricolazione}
Anche la procedura di immatricolazione avviene attraverso il sistema ESSE3, seguendo la procedura indicata dalla voce "registrazione". Dall'anno scorso la procedura di immatricolazione avviene completamente online senza che sia necessario presentarsi in segreteria o spedire documenti. Per completare correttamente la propria iscrizione è necessario caricare una foto (formato fototessera, non delle vacanze al mare! Un .jpeg 660x791 a 72 DPI è il formato consigliato.), compilare tutti i campi richiesti e infine stampare il MAV e pagare la prima rata.
La procedura è leggermente diversa per i corsi di laurea (ad esempio se sono a numero programmato, se c'è un test di ammissione) ma è accuratamente spiegata sul sito dell'università alla voce immatricolazione (http://www.unimib.it/go/45470).Per avere altri dettagli potete sempre consultate la Guida dello Studente (scaricabile dal sito dell'università) o contattatare il servizio di Orientamento d'Ateneo ( www.unimib.it/orientamento o con una mail a orientamento@unimib.it).

 

 

 

 

Medicina e Chirurgia

Presentazione
In università ci sono più di 2000 studenti di medicina che studiano tra sette corsi di laurea triennale, due corsi di laurea magistrale e due corsi di laurea magistrale a ciclo unico. 

Corsi di laurea triennale
      • Fisioterapia (63 posti) 
      • Igiene dentale (32 posti) 
      • Infermieristica (357 posti) 
      • Ostetricia (43 posti) 
      • Tecniche di laboratorio biomedico (27 posti) 
      • Tecniche di radiologia medica, per immagini e radioterapia (32 posti) 
      • Terapia della neuro e psicomotricità dell'età evolutiva (27 posti) 

Corsi di laurea magistrale
      • Biotecnologie mediche 
      • Scienze infermieristiche ed ostetriche

Corsi di laurea magistrale a ciclo unico
      • Medicina e chirurgia (135 posti) 
      • Odontoiatria e protesi dentaria (20 posti) 
I corsi di laurea magistrale a ciclo unico hanno la durata di 6 anni. 

Contatti
Sito ufficiale di medicina: http://www.medicina.unimib.it 
Rappresentanti degli studenti: http://www.medicina.unimib.it/cmsMedicina/Spazio\_ Studenti/pagina\_ Spazio\_ studenti.html 

 

 

Scienze Matematiche, Fisiche e Naturali

Presentazione
Scienze ha più di 5000 iscritti e offre 10 corsi di laurea triennale e 11 corsi di laurea magistrale.

Corsi di laurea triennale
     • Biotecnologie (225 posti) 
     • Fisica 
     • Informatica (300)
     • Matematica 
     • Ottica e optometria (150) 
     • Scienza dei materiali 
     • Scienze biologiche (225 posti) 
     • Scienze e tecnologie chimiche (100) 
     • Scienze e tecnologie geologiche 
     • Scienze e tecnologie per l'ambiente (150)

Corsi di laurea magistrale
     • Astrofisica e fisica dello spazio 
     • Biotecnologie industriali 
     • Biologia 
     • Fisica 
     • Informatica 
     • Matematica 
     • Scienza dei materiali 
     • Scienze e tecnologie chimiche 
     • Scienze e tecnologie geologiche 
     • Scienze e tecnologie per l'ambiente e il territorio 
     • Teoria e tecnologia della comunicazione 

Il corso in Teoria e tecnologia della comunicazione è tenuto in collaborazione con Psicologia. 

Studiare in Bicocca
L'accesso ai corsi triennali senza il numero programmato prevede il superamento di una prova che verte principalmente sulle conoscenze di matematica e logica. Per il primo anno, la maggioranza dei corsi di laurea triennale della Scuola di Scienze è a numero programmato. I corsi di laurea magistrale prevedono requisiti curricolari e competenze che sono specificati sul manifesto dei rispettivi corsi di laurea. 
Scienze offre inoltre dei precorsi di richiami di matematica e di metodologia dello studio universitario. Durante il primo anno, sono previsti corsi di recupero per chi non avesse superato il VPI. 

Contatti
Sito di Scienze: www.scienze.unimib.it 
Precorsi: http://www.scienze.unimib.it/?page_id=243

 

 

 

5.3 150 Ore
Tra le opportunità previste dalla legge n.390 del 1991 a sostegno del diritto allo studio, l'università ha previsto le cosiddette 150 ore ovvero collaborazioni con gli studenti per lo svolgimento di diverse attività (tutor dei laboratori informatici, supporto alle biblioteche o alle Segreterie Studenti, orientamento per le matricole, etc.) per un massimo – appunto - di 150 ore. Queste collaborazioni sono a tempo parziale e retribuite 9€ all'ora. Si accede tramite bando basato su criteri di reddito e merito, per cui è necessario essere iscritti almeno al secondo anno per potervi partecipare. Durante ogni anno accademico vengono proposti diversi bandi, sia d'Ateneo sia di Facoltà, per i quali possono essere richiesti alcuni requisiti tecnici (es. abilità informatiche di base). Se siete interessati, controllate periodicamente http://www.unimib.it/go/45056 sul quale vengono pubblicati bandi e graduatorie. 

5.4 Job Placement
Promosso dall'ufficio Job Placement, il servizio VULCANO offre la possibilità a tutti gli studenti, laureati o laureandi, di essere inseriti in un database di curricula che permette all'ufficio di favorire e perseguire l'incontro tra offerta e domanda di lavoro. Gli iscritti al servizio ricevono via mail proposte inoltrate da aziende interessate ai profili professionali proposti, sia per offerte di lavoro sia per possibilità di stage. 

5.5 Counselling psicologico
La Bicocca offre anche un utile, ma spesso sconosciuto, servizio di counselling per chi avesse problemi di studio o situazioni personali che inibiscono il corretto svolgimento della carriera universitaria. Il servizio consiste in un ciclo medio-breve di incontri individuali con uno psicoterapeuta od uno psicologo clinico specializzati ad operare con pazienti in età tardoadolescenziale e giovane-adulta. Durante il percorso lo specialista cercherà di stabilire con lo studente degli obiettivi chiari da raggiungere entro la fine della serie di incontri. 
Il servizio è totalmente gratuito, per prenotare un incontro o semplicemente avere informazioni è possibile inviare un' e-mail o prendere contatto telefonicamente. I contatti sono disponibili all'indirizzo http://www.unimib.it/go/46063.

5.6 Sport
Correre per arrivare in università non si può considerare propriamente uno sport. Meglio qualcosa di classico, divertente e, possibilmente, a buon prezzo. Per questo potete rivolgervi al CUS (Centro Universitario Sportivo), che offre corsi a prezzi convenienti, fornisce sconti ed agevolazioni per alcuni impianti comunali e organizza settimane bianche, gite ed escursioni. In Bicocca c'è una palestra convenzionata (in U12), accessibile a studenti e dipendenti,  aperta dal lunedì a venerdì dalle ore 12 alle ore 20 (dalle ore 10 alle ore 12 solo per i residenti del pensionato) che ha dei prezzi veramente imbattibili! Per informazioni www.cusmilano.it oppure rivolgetevi al CUS point in U6 al primo piano. 

5.7 Il Coro
Nel 2007 si è composto all'interno dell'università come associazione il Coro; esso è composto da una trentina di persone accomunate dalla passione per la musica classica e popolare. La partecipazione è aperta a tutti ed il Coro si incontra tutti i mercoledì dalle ore 17.30 alle ore 19.30, nell'aula 8 dell'edificio U4 a partire da mercoledì 16 settembre. Per informazioni: http://xoomer.virgilio.it/corobicocca/

5.8 Biblioteche e aree studio
La Biblioteca d'Ateneo ha un'unica gestione ma è strutturata in tre sedi (quella Centrale, di Scienze e di Medicina). La biblioteca è il luogo ideale per studiare in tranquillità; è aperta dal lunedì al venerdì dalle 9 alle 18.30. Molti sono i servizi offerti: consultazione, prestito, prestito interbibliotecario, fotocopie, reperimento di articoli di vari natura, consulenza bibliografica, spazi di studio individuali per chi è sotto tesi e poi vi è un intuitivo catalogo on line (OPAC). 
Le tre sedi sono:
      - Sede Centrale: Piazza dell'Ateneo Nuovo 1, edificio U6 II piano. Aree disciplinari: diritto, economia, informatica, psicologia, sociologia, scienze della formazione e statistica. 
      - Sede di Scienze: Piazza della Scienza 3, edificio U2 I piano (sala monografie) e piano -1 (sala periodici). Aree disciplinari: biologia, chimica, fisica, geologia e matematica. 
      - Sede di Medicina: Via Cadore 48, Monza, edificio U8 piano terra. Aree disciplinari: medicina. 
Inoltre nel campus della Bicocca è presente un'altra Biblioteca afferente al CIDiS, che si trova al secondo piano dell'edificio U12. A differenza della biblioteca d'Ateneo è gestita da un organo interunivesitario (appunto il CIDiS) . Per poter usufruire del prestito dei libri è necessario pagare una quota di 10 euro. Lo scorso anno, in via sperimentale e su richiesta degli studenti, la biblioteca del CIDiS ha prolungato il servizio di apertura, chiudendo alle 22; visto il successo dell'iniziativa, speriamo sia possibile mantenere il servizio anche per i prossimi anni. 
Altri luoghi di studio sono le Aree Studio sparse tra i vari edifici dell'Università, dove si può trovare una maggiore tranquillità rispetto ai tavoli nei corridoi o nei cortili interni.

5.9 Bar e mense
La disponibilità di servizi di mensa si compone delle seguenti strutture: 
      - la mensa principale (convenzionata CIDiS) è quella in U6 al Piano (-1) con accanto il bar; 
      - la mensa convenzionata dello studentato (primo piano U12) formata da 4 self-service indipendenti e completi; 
      - la tavola calda (non convenzionata CIDiS) al primo piano dell'U7 con accanto il bar; 
      - a Monza esiste una convenzione con la mensa dell'ospedale S. Gerardo. 
I bar, invece, si trovano: 
      - in U6 al piano terra  e al piano (-1); 
      - in U7 al primo piano; 
      - in U3 al piano (-1); 
      - in U12 direttamente sulla strada, vicino all'ingresso della mensa. 
Se avete un reddito che soddisfa i requisiti del bando C.I.Di.S. (pubblicato sul sito web.consorziocidis.it) è possibile ottenere la carta magnetica che permette di pagare i pasti a prezzi calmierati (le tariffe sono diverse a seconda della fascia di reddito cui si appartiene). 
Infine la presenza dell'università ha prodotto il moltiplicarsi di ristoranti, bar e pizzerie d'asporto; è inoltre da segnalare la presenza di 2 centri commerciali facilmente raggiungibili (uno con i mezzi, l'altro a piedi) forniti di numerosi ristoranti di vario genere. 

 

\subsection{Erasmus: studiare all'estero}
Anzitutto, che cos'è il Progetto Erasmus? L'Erasmus è il principale progetto europeo di mobilità studentesca internazionale e permette ogni anno a migliaia di studenti di muoversi per un periodo di tempo (dai tre mesi ad un anno) ed andare a studiare e dare esami in un altro paese europeo. Lo studente in Erasmus è equiparato agli studenti dell'università ospitante e avrà quindi accesso a tutti i servizi offerti loro, oltre ad alcuni servizi specifici come l'aiuto nella ricerca dell'alloggio o i corsi di lingua. Per partire è necessario consultare i bandi che ogni anno la propria facoltà pubblica, solitamente in febbraio/marzo dell'anno precedente a quello che vi interessa, e fare domanda attraverso gli appositi uffici (la procedura completa è descritta all'interno del bando di riferimento). 
Ogni facoltà propone le destinazioni con cui è convenzionata. Da due anni grazie al lavoro dei nostri rappresentanti in CdA la Bicocca stanzia i fondi per le borse prima dell'inizio dei periodi di studio all'estero, quindi possiamo già dirvi che chi partirà per l'Erasmus nell'A.A.12/13 riceverà 300€ ogni mese. Proveremo l'anno prossimo a chiedere alla Bicocca di anticipare ancora la delibera in modo che si sappia già a febbraio (quando vanno presentate le richieste) a quanto ammonterà la borsa, che in ogni caso non dovrebbe ridursi rispetto agli anni precedenti. Fino allo scorso anno si aspettava lo stanziamento ministeriale e si potevano approvare le borse solamente a Febbraio/Marzo, cioè un anno dopo il termine per presentare le domande e quando molti studenti erano già partiti. Anche il CIDiS (Consorzio Interuniversitario per il Diritto allo Studio) bandisce delle borse per gli studenti in partenza per università estere assegnate in base al reddito e ai punti di credito conseguiti negli anni. Possono partire tutti gli studenti iscritti ad un anno successivo al primo (il primo anno si può fare domanda per partire il secondo). Prima di partire è necessario stipulare un learning agreement, ovvero un elenco degli esami che si intendono sostenere all'estero, per avere la certezza, al ritorno, che tutti gli esami sostenuti siano riconosciuti e quali saranno, per esempio, i crediti e il voto attribuiti ad ognuno (non tutti i paesi usano sistemi con crediti e voti in trentesimi).

\subsubsection{Lingue}
L'università di partenza può richiedere la conoscenza della lingua del paese di destinazione che avete scelto, possibilmente certificata da un diploma. Altrimenti è comunque sufficiente aver passato il test di conoscenze linguistiche di inglese (o della lingua del paese di destinazione) di Ateneo. Di solito, inoltre, per le lingue meno conosciute, le Università di destinazione organizzano dei corsi specifici destinati agli studenti Erasmus.
Per un contatto diretto con l'ufficio della Bicocca che si occupa degli Erasmus, scrivete a: international.office@unimib.it

\subsection{Altri programmi di scambio}
\subsubsection{Erasmus mundus}
Interessante anche questo progetto, che offre la possibilità di studiare all'estero, non durante il proprio percorso accademico curricolare, bensì attraverso un master. 

\subsubsection{Extra}
Nuovo programma dell'Università degli Studi di Milano-Bicocca di mobilità che si rivolge a chi ha quasi concluso il proprio percorso di studi. Consente lo svolgimento di un periodo di studio all'estero finalizzato alla preparazione della tesi di Laurea Specialistica/Magistrale, della durata minima di 3 mesi e massima di 6, presso università o centri di ricerca con i quali siano già attivi contatti e/o iniziative di collaborazione accademica o scientifica con l'Università Bicocca. 
Il bando richiede la presentazione delle domande in tre scadenze quadrimestrali, attraverso le quali illustrare il proprio progetto. Nell'ambito di tale Programma, sono stati finanziati dei premi di studio da parte della Fondazione Cariplo dell'importo mensile lordo pari a 750€. 

\subsubsection{Exchange}
Con il Programma Exchange è possibile fare un'esperienza di studio in paesi europei ed extra europei presso uno dei Partners Exchange di Ateneo, per un periodo che può andare da un minimo di alcune settimane ad un anno, durante il quale studiare e dare esami che saranno riconosciuti nel piano di studi ai fini della laurea. L'Exchange è un'ulteriore possibilità di mobilità a cui può partecipare anche chi ha già fatto l'esperienza dell'Erasmus. I bandi Exchange escono verso marzo. 

\subsubsection{Summer school}
Le Summer Schools offrono la possibilità di andare in un'università estera per frequentare corsi estivi di approfondimento su tematiche o settori del proprio corso di laurea, della durata di alcune settimane. I bandi di partecipazione si possono trovare sul sito di ateneo ma anche (e soprattutto) sui siti delle facoltà che li organizzano. I corsi seguiti durante la Summer School possono essere riconosciuti come CFU a scelta, ma solo su richiesta dello studente e comunque in questo caso non è previsto un accordo tra l'università di appartenenza e quella straniera. 

Capitolo 6: Milan, l’è on gran Milan
La città di Milano è grande e caotica, affollata da pendolari, turisti, studenti, immigrati, impiegati, curiosi, muratori, operai, ragazzi etc (quasi) tutti rigorosamente... di fretta!Qualsiasi cosa stiate cercando, è molto probabile che la troviate: dalla festa latinoamericana al Cenacolo di Leonardo da Vinci, dai concerti di Vasco agli aperitivi culturali, dagli aperitivi prima di cena ai kebabbari aperti fino a notte fonda.Cercheremo di darvi un'idea molto, molto, generale di tutte le opportunità che vi si presentano.

6.1 Fare i turisti
I luoghi di interesse artistico e culturale a Milano sono numerosi.Alcuni sono noti in tutto il mondo, come il Cenacolo, il Duomo, il Castello Sforzesco, la Pinacoteca di Brera, la basilica di Sant'Ambrogio, il teatro alla Scala. Altri sono più nascosti ma altrettanto interessanti come l'Anfiteatro Romano vicino a San Lorenzo alle Colonne, luogo di ritrovo per i giovani; la Chiesa di Santa Maria presso San Satiro, con il finto coro progettato dal Bramante; piazza dei Mercanti, dove parlando in un angolo delle colonne sentirete la vostra voceamplificata dalla parte opposta; l'insolita Torre Velasca...
Il consiglio migliore è quello di prendere la bicicletta (se avete il coraggio di lanciarvi nel traffico cittadino) o il tram e girare senza meta scoprendo la città. Se siete amanti delle due ruote, vi consigliamo il giovedi sera i Critical Mass, giri in bicicletta per Milano con sconti i determinati pub. Quando sarete stanchi, vi consigliamo una sosta in uno dei (purtroppo pochi) parchi rimasti in città, i più famosi sono il Parco Sempione, il Parco di Porta Venezia e parco Forlanini, senza escludere il Parco Nord a pochi minuti dalla nostra università. Per le giornate di pioggia, non mancano i musei e le mostre. Ce ne sono per tutti i gusti: Planetario, Acquario Civico, Museo di Storia Naturale, Museo della Scienza e della Tecnica, la Pinacoteca Ambrosiana, la Triennale (con la sede del design anche in Bovisa) e il Pac, per gli amanti dell'arte moderna. Di recentissima apertura infine il Museo del 900, vicino a Palazzo Reale e un insolito Wow, museo del fumetto aperto negli ultimi mesi.

6.2 La capitale della moda
Non ditelo troppo in giro, ma a pochi milanesi capita di fare shopping nel celebre quadrilatero della moda, tanto meno agli studenti, solitamente squattrinati. Prezzi più abbordabili e negozi meno esclusivi si trovano in via Torino, lungo Corso Vittorio Emanuele o in Corso Buenos Aires; ma state attenti che il sabato pomeriggio sono presi d'assalto e vi passerà presto la voglia di spendere i vostri soldi. Se cercate qualcosa di più particolare vi consigliamo la Fiera di Senigallia, appuntamento ogni sabato nella zona di Porta Genova per cercare tra bancarelle di usato e non oggetti fuori dal normale (troverete anfibi, giacche militari, dischi in vinile, fumetti, vestiti colorati, magliette del Che, oggetti da giocoleria).
Quando si avvicinano le feste invece, non potrete perdervi la fiera degli Oh bej! Oh bej! e quella dell'Artigianato, tradizionali appuntamenti durante il ponte di Sant'Ambrogio per milanesi in cerca di regali natalizi.

6.3 Cinema e teatri
L'offerta cinematografica in città è ovviamente vasta e varia. Durante il mese di settembre, la stagione viene aperta dal "MilanoFilmFestival" che propone oltre alle proiezioni di corti e lungometraggi, numerosi eventi che animano la città durante tutta la durata del festival. A concluderla invece nel mese di giugno la rassegna "Cannes e dintorni" che propone (spesso in anteprima e in lingua originale) i film che hanno partecipato al celebre Festival francese.Un appuntamento particolare è il "Festival MIX di cinema gaylesbico e queer culture", che propone oltre a corti, lungometraggi, presentazioni di libri, dibattiti e incontri sempre su questa tematica.
In città si trovano numerose sale cinematografiche e multisale, in cui potrete trovare sia le ultime uscite sia iniziative come cineforum, film d'autore, rassegne a tema. Vi ricordiamo che il mercoledì tutti i biglietti hanno un costo ridotto, vi consigliamo anche di controllare particolari offerte (come la promozione "Ricomincio da tre" degli Uci Cinema che offrono ogni martedì un film a 3 euro) o riduzioni per gli studenti.In città sono presenti numerosi teatri, molti offrono riduzioni per studenti o campagne abbonamento che permettono di vedere più spettacoli a prezzi agevolati, vi consigliamo il sito www.lombardiaspettacolo.com per tenerli tutti sotto controllo.

6.4 Dove si va stasera?
Un appuntamento classico per gli universitari milanesi è il rito dell'Happy Hour,  servizio offerto ormai da quasi tutti i locali, durante l'orario di cena al prezzo di un cocktail potrete mangiare a volontà da un (solitamente) ricco buffet. I costi variano dai 5 ai 15 euro, quelli con un rapportoqualità/prezzo più conveniente sono, secondo noi: il Maga Furla (vicino all'università e comodo se finite tardi), il Ciu's in via Spontini (frequentato da molti studenti erasmus), il Blender Bar in piazzale Susa (sempre pieno ma con un buffet degno dei migliori pasti luculliani), l'Hora Feliz vicino a San Lorenzo (locale piccolo, ma con numerosi tavoli in strada e cibo ottimo in gran quantità) e, sempre per restare in zona, l'Yguana. Se volete una cena con calma, vi consigliamo le guide prodotte da Terra di Mezzo: appaMilano (che presenta una rassegna tra i più buoni e convenienti ristoranti milanesi) e PappaMondo (guida completa, precisa e pratica di tutti i ristoranti etnici presenti a Milano).
Anche per vedersi dopocena non mancano i locali dove trovarsi. Le zone più frequentate per una passeggiata o un cocktail con gli amici sono quelle di Brera, dei Navigli e delle colonne di San Lorenzo; mentre in Corso Como e vicino al Parco Sempione troverete i locali più costosi e le discoteche più selettive. Si possono trovare posti interessanti anche in zone più lontane dal centro, come il Frida (via Pollaiuolo 3), una vecchia fabbrica ristrutturata in giardino, enoteca, art gallery, cocktail bar e molto altro; oppure il Turnè (via Paolo Frisi 3) piccolo locale in zona Porta Venezia ma con una programmazione diversa e varia ogni sera della settimana (dallo spritz a 3 euro il martedì al cinema domenicale, passando per degustazioni di vini e offerte speciali per i cocktail). In estate non si può mancare all'appuntamento con il Carroponte a sesto San Giovanni che da qualche anno offre concerti a prezzo popolare all'aperto. 
Se siete amanti della birra, non potete perdervi il Birrificio Lambrate (via Adelchi 3) che offre una selezione di birre di propria produzione; l'appuntamento il giovedì sera è in zona centrale per l'offerta di due medie a 5 euro dell'Outback (via Carlo Tenca 10) mentre la domenica e il lunedì l'EastEnd vicino al Cimitero di Lambrate vi offre una pinta al costo della piccola. Per essere sicuri di non perdervi nessun appuntamento e rimanere aggiornati sulle nuove aperture e sulle offerte speciali (tra cui molte rivolte alla clientela universitaria) vi consigliamo la guida Zero (distribuita gratuitamente ogni quindici giorni in locali e negozi) e gli inserti settimanali di alcuni giornali, come ViviMilano del Corriere o TuttoMilano della Repubblica. 

6.5 Centri sociali e circoli Arci
Il più famoso centro sociale milanese è il Leoncavallo, in via Watteau, offre durante tutto l'anno concerti, incontri, proiezioni cinematografiche, feste, corsi di lingua, laboratori (per maggiori informazioni www.leoncavallo.org). Vi consigliamo vivamente di partecipare a la Terra trema, a novembre con degustazioni di vini da tutta Italia, e la festa del raccolto e della semina..Leggendo i manifesti sui muri della città o i giornali, scoprirete che in quasi ogni quartiere si trova un vecchio edificio abbandonato, ora occupato e autogestito da collettivi che propongono manifestazioni ed eventi; ad esempio il Casa Loca, vicino alla Bicocca, dove vi offriranno un piatto di pasta a 3 euro e partite al calcetto infinite e la Cascina Torchiera in zona Certosa.
I circoli Arci presenti tra città e provincia sono più di un centinaio, si tratta di Associazioni con uno luogo di ritrovo dove solitamente si trovano cibo e bevande a prezzi molto convenienti, si organizzano concerti (spesso di artisti emergenti o "non convenzionali"), è possibile partecipare a dibattiti, cineforum e molto altro. 
Per entrare è necessario avere la tessera (il cui costo è attorno ai 10 euro) che ha validità annuale e permette l'ingresso in tutti i circoli Arci d'Italia.Potete consultare il sito www.arcimilano.it per avere l'elenco completo, vi segnaliamo il circolo Magnolia (via Circonvallazione-Segrate) dove si organizzano concerti e serate che attirano giovani da tutta la provincia, il circolo Metissage (quartiere Isola, dietro la stazione di Porta Garibaldi) se volete godervi una serata tranquilla in compagnia della buona musica e l'Agorà, circolo dell'hinterland milanese che offre spettacoli alternativi e di gruppi emergenti.  

6.6 Trasporto pubblico
Il trasporto pubblico milanese non è all'altezza di quello delle altre grandi città europee, ma si difende con dignità ed è tutt'ora oggetto di continue modifiche e miglioramenti (speriamo). L'A.T.M. gestisce quattro linee di metrò che collegano le periferie con il centro della città, numerosi autobus e tram che coprono in modo più capillare il territorio urbano ed extraurbano. Inoltre il cosiddetto "passante", ovvero le linee suburbane di Trenord, agevola gli spostamenti verso comuni leggermente più distanti. 
Il biglietto urbano costa 1,50 euro (valido per un'ora e un quarto per spostamenti interni alla città e una sola corsa in metro), ma se siete assidui frequentatori dei mezzi pubblici potrebbe risultarvi più conveniente l'abbonamento mensile (17 euro) o quello annuale (170 euro) riservato agli studenti fino ai 26 anni. Per maggior comodità, potrete acquistare un biglietto giornaliero (4,50 euro per 24 ore) o un carnet da dieci viaggi (utilizzabili in momenti diversi) o potreste provare la nuova tessera RicaricaMi su cui caricare biglietti, settimanali o carnet da avere sempre a portata di mano nel portafoglio.
I mezzi pubblici sono comodi e veloci durante il giorno per evitare il traffico ed il parcheggio a pagamento. In settimana l'ultima corsa della metropolitana è verso mezzanotte, mentre da quasi due anni il servizio notturno per il week end è stato finalmente potenziato con autobus sostitutivi delle linee metropolitane che girano tutta la notte e nuove linee notturne; un servizio alternativo è però quello del Radiobus, un bus prenotabile per spostarsi da dove a dove volete fino alle 2 di notte.
Per maggiori informazioni sui servizi, sul costo dei biglietti, per poter calcolare il percorso più breve e quant'altro potete consultare il sito www.atm-mi.it o chiamare il numero verde 800 80 81 81. 

\mysec{Contatti}
Puoi trovare le aule dei rappresentanti degli studenti  nell'edificio U2 (Piazza della Scienza - Fisica) al piano terra, tel. 02.6448 2061 oppure nell'edificio U6 (Piazza dell'Ateneo NuovoEconomia,Psicologia, Scienze della Formazione, Giurisprudenza, Rettorato) al piano -1, tel. 02.6448 6992
Se vuoi contattarci puoi farlo anche via e-mail all’indirizzo listedisinistra@gmail.com oppure visitando il sito www.ListediSinistra.org dove ti puoi iscrivere alla nostra newsletter utilizzata per aggiornamenti riguardo iniziative e incontri o per ricordare scadenze accademiche e amministrative.
Da quest'anno siamo anche su facebook all'indirizzo http://www.facebook.com/ListediSinistra 
Non ti basta? Se hai un problema relativo ad uno specifico Corso di Studi, ecco qui a chi puoi rivolgerti:

Scienze:
\begin{itemize}
\item Matematica: n.grittini@campus.unimib.it (Nicola Grittini)
\item Biotecnologie: m.gnugnoli@campus.unimib.it (Marco Gnugnoli)
\item Informatica: rappresentantistudenti@disco.unimib.it
\item Fisica: e.panontin@campus.uimib.it (Enrico Panontin)
\item Scienze e tecnologie per l'ambiente: s.calabrese4@campus.unimib.it (Silvia Calabrese)
\end{itemize}   
Sociologia: 
\begin{itemize}
\item t.pettinato@campus.unimib.it (Thomas Pettinato) oppure rappresentanti.sociologia@gmail.com
\end{itemize}
Giurisprudenza:
\begin{itemize}
\item s.zanco@campus.unimib.it (Sebastiano Zanco)
\end{itemize}   
Medicina: 
\begin{itemize}
\item d.celsi@campus.unimib.it (Davide Celsi)
\end{itemize}   
Statistica: 
\begin{itemize}
\item a.torti7@gmail.com (Andrea Torti)
\end{itemize}
Economia: 
\begin{itemize}
\item Ecoban: g.papaleo92@gmail.com (Giuseppe Papaleo)
\item Ecocom: s.tatti1@campus.unimib.it (Simone Tatti)
\item Ecofin: n.digravio@gmail.com
\end{itemize}

\end{multicols}
%\printindex

\end{document}

