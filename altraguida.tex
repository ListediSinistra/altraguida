\documentclass[12pt,a4paper,twoside]{article}
\usepackage[T1]{fontenc}
\usepackage[utf8x]{inputenc}
\usepackage[italian]{babel}
% \usepackage[utf8x]{inputenc}
\usepackage{amsmath,amssymb,amsfonts,textcomp}
\usepackage{color}
\usepackage{calc}
\usepackage{eurosym}
\usepackage[dvips]{geometry}
\usepackage{eso-pic}
\usepackage{graphicx}
%\usepackage{makeidx}
\usepackage{fancyhdr,fancybox}
%\usepackage[none,light,timestamp]{draftcopy}
\usepackage{multicol}
\usepackage{enumitem}
\usepackage{url}
\usepackage[breaklinks=true]{hyperref}
%%%%%%%%%%
%%%%%%%%%% 
\usepackage{sectsty}
\usepackage{ifthen}
\usepackage{sectionbox}
% \usepackage{skull}
\usepackage{lettrine}

\doublesectionbox
\shadowsubsectionbox
\shadowsubsubsectionbox


\definecolor{sectboxrulecol}{rgb}{0,0,0}
\definecolor{sectboxfillcol}{rgb}{0,0,0}
\definecolor{sectboxtextcol}{rgb}{1,1,1}

\definecolor{subsectboxrulecol}{rgb}{0.75,0.75,0.75}
\definecolor{subsectboxfillcol}{rgb}{0.9,0.9,0.9}
\definecolor{subsectboxtextcol}{rgb}{0,0,0}

\definecolor{subsubsectboxrulecol}{rgb}{0.75,0.75,0.75}
\definecolor{subsubsectboxfillcol}{rgb}{0.9,0.9,0.9}
\definecolor{subsubsectboxtextcol}{rgb}{0,0,0}

\newcommand{\mysecname}{INDICE}
\newcommand{\mysec}[1]{%\null\vskip24pt%
\gdef\mysecname{\uppercase{#1}}
%\renewcommand{\mysecname}{\uppercase{#1}}
\begin{sectionbox}{#1}\relax
\end{sectionbox}}
\sectionfont{\sffamily}%\textcolor[gray]{0.2}}

\newcommand{\mysubsecname}{\relax}
\newcommand{\mysubsec}[1]{%\null\vskip12pt%
%\renewcommand{\mysubsecname}{#1}
\gdef\mysubsecname{#1}
\relax\vskip6pt \begin{subsectionbox}{#1}\relax
\end{subsectionbox}}
\subsectionfont{\sffamily}%\textcolor[gray]{0.2}}

\newcommand{\mysubsubsec}[1]{%\null\vskip6pt%
\relax\vskip6pt \begin{subsubsectionbox}{#1}\relax
\end{subsubsectionbox}}
\subsubsectionfont{\sffamily}%\textcolor[gray]{0.2}}

\makeatletter\renewcommand\paragraph{\@startsection{paragraph}{4}{\z@}%
                                    {12pt \@plus1ex \@minus.2ex}%
                                    {-1em}%
                                    {\normalsize\bfseries\sffamily}}
\makeatother         

%\paragraphfont{\sffamily}
                           
%%%%%%%%%%
%%%%%%%%%% 


\geometry{top=0.6in,bottom=0.7in,left=0.7in,right=0.7in}
%\geometry[E]{top=0.7in,bottom=0.7in,left=0.7in,right=0.6in}
%\hypersetup{colorlinks=true, linkcolor=blue, filecolor=blue, pagecolor=blue, urlcolor=blue}
%\draftcopySetGrey{0.90}
%\draftcopyName{Bozza}{150}


%\newcommand{\collegamento}[2]{\catcode`\%11\relax {\href{#1}{\underline{#2}}}\catcode`\%14\relax}

%\newcommand{\collegamento}[2]{\href{#1}{\underline{#2}}}

\newcommand{\collegamento}[2]{\href{#1}{\tt #2}}

\newcommand{\collegamail}[1]{\collegamento{mailto:#1}{#1}}

\newcommand{\gdstud}{ \collegamento{http://www.unimib.it/upload/bicocca\%20guida\%20\%202007\%20-\%20INTERNET\%20-.pdf}{Guida dello
Studente}}

%\newcommand{\LdS}{{\textbf{Listedi}}{\textbf{\textcolor[rgb]{0.8,0.0,0.2}{Sinistra}}}}
%\newcommand{\LdS}{Listedi\textcolor[rgb]{0.8,0.0,0.2}{Sinistra}}
\newcommand{\LdS}{{\textbf{Listedi}}{\textbf{\textcolor[rgb]{0.3,0.3,0.3}{Sinistra}}}}

\newcommand{\tab}{\hspace{5mm}}

\newcommand\mynote[1]{\index{aaa@\textbf{Tom:}}\marginpar%
{\raggedright\rule{.08\marginparwidth}{12pt}\vbox{\raisebox{-11pt}{\hbox{\hskip 2pt \scriptsize{\sl Tom:}}}\\\rule{.92\marginparwidth}{.4pt}}\\ \footnotesize{#1}}
}%

\newcommand{\postit}[1]{\textcolor[rgb]{0,0.8,0.2}{#1}}

\newcommand\NdP[1]{\index{aaa@\textbf{le note di Pietro sono alle pagg.:;}}\marginpar%
{\raggedright\rule{.08\marginparwidth}{12pt}\vbox{\raisebox{-11pt}{\hbox{\hskip 2pt \scriptsize{\sl NdP:}}}\\\rule{.92\marginparwidth}{.4pt}}\\ \footnotesize{#1}}}%
\renewcommand\NdP[1]{\relax}

\newcommand{\boldsansserif}[1]{\textsf{\textbf{#1}}}

%\newcounter\warningcounter
%\setcounter{\warningcounter}{1}
%\newcommand{\warning}[2]{\setcounter\warningcounter#2\fbox{{\ifnum\warningcounter=3 \skull\skull\skull \elseif \ifnum\warningcounter=1 \skull\skull \fi\else \skull \fi} \textbf{ATTENZIONE:}#1}}

%\newcommand{\mywarning}[1]{\noindent\ovalbox{\hskip12pt\begin{minipage}{0.85\columnwidth}$\skull\skull\skull$ \textbf{ATTENZIONE!}\\#1\end{minipage}\hskip12pt}}

\newcommand{\mywarning}[1]{\noindent\ovalbox{\hskip12pt\begin{minipage}{0.85\columnwidth}\lettrine[image=true, lines=2, loversize=0.25]{warn.pdf}{} \textbf{ATTENZIONE!}\\#1\end{minipage}\hskip12pt}}


\newcommand{\vedi}[1]{(vedi p.~\pageref{#1})}

\setlist{noitemsep, topsep=0pt, leftmargin=15pt, labelindent=0pt}
\setlength{\headheight}{15pt}
\setlength{\columnsep}{24pt} 
\setlength{\columnseprule}{0.4pt}   
\setlength{\columnwidth}{10cm}


\pagestyle{fancy}
%\newcommand{\chaptermark}[1]{\markboth{#1}{}}
%\renewcommand{\sectionmark}[1]{\markright{\thesection\ #1}}
\fancyhf{}
\fancyhead[LO]{\sf\bfseries\thepage \quad www.\LdS.org}
\fancyhead[RE]{\LdS @gmail.com \quad \sf\bfseries\thepage}
%\fancyhead[LO]{\sf\bfseries\rightmark}
\fancyhead[LE]{\sf\bfseries \relax\quad\thesubsection{} \quad \mysubsecname}
%\fancyhead[RE]{\sf\bfseries\leftmark}
\fancyhead[RO]{\sf\bfseries \ifnum\thesection=0 \relax \else \thesection \fi{}\quad \mysecname\quad}
\marginparwidth45pt
%\fancyfoot[C]{\includegraphics[height=9mm, keepaspectratio]{logo.pdf}}

\makeindex

%qui si definisce lo sfondo da richiamre con un \AddToShipoutPicture{\Sfonda}
%si disattiva con \ClearShipoutPicture - si veda pacchetto eso-pic -
%\include{sfonda}


\hyphenation{e-ster-no in-di-spen-sa-bi-le tra-sfe-ri-men-ti spe-cia-li-sti-ca ma-tu-ra-ti de-sti-na-zio-ne bru-sco cre-di-ti o-spe-da-li pos-si-bi-li-t\'a i-ni-zio neu-ro-psi-co-lo-gi-a i-nol-tre acqui-sto Se-sto in-va-li-di-t\`a auto-va-lu-ta-zio-ne a-spet-ti sod-di-sfi o-gni Ni-guar-da E-di-fi-cio li-vel-lo luo-ghi o-biet-ti-vi fun-zio-na-men-to con-si-de-ra-zio-ne li-ber-t\'a ap-pro-fit-ta mi-ni-ste-ro in-te-res-se cul-tu-ra-le per-met-ten-do e-ste-ro i-sti-tu-zio-ne ri-chie-sta bre-scia ber-ga-mo e-ma-na-ti li-bri bi-blio-te-che i-nol-tre que-sta in-du-stri-e e-du-ca-ti-ve du-re-ran-no ri-ser-va-te gra-dua-to-ria me-ri-to e-so-ne-ri so-cie-t\'a i-scri-zio-ne bi-blio-te-ca o-rien-ta-men-to or-ga-niz-za-ti-vo re-gi-stra-ti po-ter re-qui-si-ti in-se-ri-ti ri-ce-ve-ran-no que-stio-na-rio in-di-vi-dua-re di-sa-bi-li i-ni-zia-ti-ve a-scol-ta-re pro-ble-ma-ti-che cam-bia-men-to am-mi-ni-stra-zio-ni per-so-na-le scri-vi-ci per-met-to-no si-tu-a-to fa-ci-li-ta-re rap-pre-sen-tan-ti bio-lo-gi-che cur-ri-cu-la ma-gi-stra-le o-biet-ti-vo ca-rat-te-riz-za-zio-ne geo-lo-gi-che af-fac-ciar-si ca-len-da-rio scien-ti-fi-co ma-na-ge-ment sta-ti-sti-ca col-la-bo-ra-zio-ne bi-so-gno-si co-sti-tu-i-sco-no i-ni-zia-ti-va me-dian-te si-tua-zio-ni tes-se-rar-si ri-sto-ran-te qua-li-t\'a so-li-da-le se-ra-te na-vi-gli-o e-stre-ma-men-te in-te-res-san-tis-si-me o-ri-gi-na-le ras-se-gne ca-po-la-vo-ri spin-go-no e-si-to mec-ca-ni-smi pre-oc-cu-pa-te-vi  di-scu-to-no e-di-fi-cio ma-te-ma-ti-ca o-biet-ti-vo la-bo-ra-to-ri si-gni-fi-ca-ti-va me-di-ci-na a-na-li-si si-gni-fi-ca e-sem-pi-o chi-a-ma-to que-sti-o-ni glo-ba-le a-zi-en-da-li-sti-co in-ter-na-zio-na-li ge-ne-ra-li-sta at-ti-vi-t\'a e-du-ca-zio-ne la-bo-ra-to-rio ca-rat-te-riz-za rap-pre-sen-tan-ti fun-zio-na-li cia-scun co-min-cia-no}

\begin{document}
%\AddToShipoutPicture{\Sfonda}
\sf  %font computer modern sans serif 
%
\pagenumbering{roman}
\title{L'\emph{altra}\/guida\footnote{{\sf versione 2.2 del \today}}}
\author{realizzata da \LdS
\footnote{\textsf{grazie al contributo derivante dal bando per le attivit\`a delle associazioni
studentesche dell'Universit\`a degli Studi di Milano--Bicocca \collegamento{http://www.unimib.it}{www.unimib.it}.\newline \null \newline}.}\\ 
Rappresentanti degli Studenti \\ \collegamento{http://www.listedisinistra.org}{www.\LdS.org}\\ 
\collegamento{mailto:listedisinistra@gmail.com}{\LdS@gmail.com}}
\date{Anno Accademico PROVA}

\maketitle

\thispagestyle{empty}

\vskip.5in
\begin{multicols}{2}
\tableofcontents
\end{multicols}
\vskip6pt \hrule \vskip6pt

\begin{multicols}{2}
\pagenumbering{arabic}
\include{capitoli/capitolo1}
\include{capitoli/capitolo2}
\include{capitoli/capitolo3}
\include{capitoli/capitolo3-2}
\mysec{Studiare in Bicocca}
La Bicocca offre per l'anno accademico 2013/14 un'ampia scelta di corsi. L'offerta formativa conta infatti 32 corsi di laurea, 34 corsi di laurea magistrale e 4 corsi di laurea magistrale a ciclo unico. I corsi di studio offerti dall'Università sono riconducibili ad otto aree tematiche, sovrapposte alle vecchie facoltà che  con l'introduzione delle legge Gelmini e del nuovo statuto di ateneo cessano di esistere; queste aree sono: Economia, Giurisprudenza, Medicina e Chirurgia, Psicologia, Scienze della formazione, Scienze matematiche fisiche e naturali, Scienze statistiche e Sociologia. 

\subsection{Accesso ai corsi}
Per i corsi di laurea triennale e a ciclo unico della facoltà di Medicina e Chirurgia e per Scienze della Formazione Primaria, il numero di iscritti è stabilito dal MIUR a livello nazionale. Sono previste prove di ammissione con programmi e date di svolgimento uguali in tutta Italia. Per altri corsi di studio il numero di posti è stabilito annualmente da ciascun ateneo.\\
Per i corsi di studio ad accesso libero, invece, è prevista una prova obbligatoria di valutazione alla preparazione iniziale (VPI). La prova ha lo scopo di verificare se la preparazione acquisita durante il percorso scolastico delle scuole superiori sia adeguata ai prerequisiti disciplinari di base fissati dal corso di laurea prescelto. 

\subsection{Lezioni e laboratori}
I corsi di studio sono strutturati in crediti formativi universitari (CFU), l'unità di misura dell'impegno medio di uno studente. Ogni anno di corso comporta l'acquisizione di 60 CFU, ripartiti principalmente in insegnamenti e laboratori. Ogni insegnamento può essere suddiviso in ore di lezione frontale e ore di esercitazione. I laboratori, al contrario, non prevedono lezioni frontali, e di norma è prevista una frequenza obbligatoria al 75 per cento. Gli insegnamenti e i laboratori sono incastrati nel piano di studi. Capire quali corsi sono obbligatori e quali opzionali dovrebbe essere il primo compito della matricola. Inoltre, proseguendo con i semestri, lo studente potrà correggere il piano, avendo acquisito nuove conoscenze e, magari, nuovi interessi. 

subsection{Piani di studi}
Il piano di studi è l'elenco degli esami che lo studente deve sostenere per conseguire la laurea, comprendendo sia quelli obbligatori per ogni corso di laurea sia quelli a scelta dello studente. Il piano di studi viene compilato tramite le Segreterie On Line seguendo la procedura indicata, in un periodo stabilito annualmente dall'Ateneo. Non è necessario compilarlo ogni volta, se non si ha intenzione di modificare quello dell'anno passato. \\
Ogni Corso di Laurea ha un regolamento che stabilisce il numero di esami, la tipologia e il numero di crediti da acquisire; quindi per avere un piano di studi in regola controllate le norme del vostro corso. Per maggiori indicazioni su come si compili un piano di studi, consultate la pagina del sito d'Ateneo o della vostra Facoltà. I piani di studi che non dovessero essere approvati possono essere ripresentati l'anno successivo o modificati quando si consegna la domanda di laurea.

\subsection{Esami} 
Dopo aver frequentato i corsi, arriva il momento di sostenere l'esame e portare a casa un voto, con l'annesso bagaglio di crediti. Le sessioni di esame sono tipicamente tra gennaio e febbraio e tra giugno e settembre, dopo il termine delle lezioni, ma può esserci una grande variabilità a seconda del corso di studio. L'esame può essere scritto o orale. \\
ATTENZIONE: Da regolamento, anche quando l'esame è scritto è previsto un momento di confronto con il docente, in cui potete visionare il compito.\\ 
Alcuni esami sono definiti dalla facoltà propedeutici ad altri. Questo significa che per potersi iscrivere all'esame del corso B, e sostenere l'esame, viene richiesto il superamento dell'esame A. Si presti quindi molta attenzione al regolamento del corso di studi, dove sono indicate tutte le propedeuticità.

\subsection{Inglese e informatica}
Tutti i corsi di studio prevedano prove di conoscenza della lingua inglese (o di un'altra lingua dell'Unione Europea) e di informatica. Non tutti i corsi prevedono entrambe le prove: i corsi dell'area scientifica, per esempio, non richiedono quella di informatica.Le prove di lingua e di informatica sono propedeutiche a tutte le attività del secondo anno. In altre parole, non sarà possibile sostenere esami del secondo anno senza averle prima sostenute. Un buon consiglio è quello di levarsi il pensiero al più presto, approfittando delle conoscenze fresche di scuola superiore. 

\subsection{Trasferimenti}
Le regole specifiche per il trasferimento da un corso di studio, anche di un altro ateneo, a un altro sono determinate dai regolamenti dei singoli corsi di studio. Riguardo al riconoscimento dei crediti già acquisiti, il regolamento studenti garantisce il riconoscimento di almeno il 50 per cento dei CFU per corsi di studio afferenti alla stessa classe (per esempio, tutti i corsi di laurea che si chiamano "fisica"). In ogni caso, è  sempre meglio prendere contatto con il referente del corso di studio di arrivo. 

\subsection{Volete saperne di più?}
Riguardo l'accesso ai corsi e l'offerta formativa: http://www.unimib.it/go/183871592 
Per le prove di lingua e di informatica: http://www.didattica.unimib.it 
Raccolta di informazioni esaustive sui corsi: http://www.unimib.it/go
Ovviamente, potete trovarci e conoscerci nelle aulette rappresentanti al piano -1 dell'U6 (di fianco ai distributori automatici) e al piano 0 dell'U2. 

\mysec{Economia}
\subsection{Presentazione}
L'università offre 4 corsi di laurea triennale nell'ambito delle discipline economiche, l'iscrizione ai quali è preceduta da un test di valutazione che, se non superato, permette comunque l'iscrizione al corso di laurea ma non l'accesso agli esami. Sono presenti inoltre quattro corsi di laurea magistrale, due dei quali hanno dei requisiti minimi di voto/media ponderata per l'accesso, che avviene solo dopo un colloquio.
\subsection{Corsi di laurea triennale}
\begin{itemize}
\item Economia delle banche, delle assicurazioni e degli intermediari finanziari. 
\item Economia e amministrazione delle imprese. 
\item Economia e commercio. 
\item Marketing, comunicazione aziendale e mercati globali.
\end{itemize}
\subsection{Corsi di laurea magistrale}
\begin{itemize}
\item Economia e Finanza. 
\item Scienze dell'economia. 
\item Scienze Economico-Aziendali. 
\item Marketing e Mercati Globali.
\end{itemize}
Per i corsi di laurea magistrale in Scienze Economico-Aziendali e Marketing e Mercati Globali l'iscrizione è condizionata dal soddisfacimento di requisiti minimi. Per il primo è necessario un voto di laurea triennale pare a 91/110 e/o una media ponderata di 21/30, ciò permette l'accesso al colloquio in base al quale è stilata una graduatoria per l'accesso. Per il secondo corso invece è necessario un voto di laurea triennale minimo pari a 94/110 e/o una media ponderata del 24/30. Anche in questo caso il soddisfacimento di questi requisiti permette l'accesso ad colloquio ed una valutazione in ingresso. Nel caso non si possedessero i requisiti gli studenti che volessero comunque iscriversi a questi corsi di laurea dovranno sostenere un test composto di dieci domande, dovendo ottenere 6/10 di risposte corrette; superato il test sarà possibile l'accesso al colloquio ed alla graduatoria. 
\subsection{Contatti}
Informazioni più dettagliate sono disponibili sul sito: http://www.economia.unimib.it/ 

\mysec{Giurisprudenza}

\subsection{Presentazione}
L'università offre un corso di laurea triennale, un corso di laurea magistrale e un corso di laurea magistrale a ciclo unico, tutti ad accesso libero nell'ambito delle scienze giuridiche.
Corsi di laurea triennale
\begin{itemize}
     \item Corso di Laurea Triennale in Scienze dei Servizi Giuridici 
Corsi di laurea magistrale a ciclo unico
     \item Corso di Laurea Magistrale a ciclo unico in Giurisprudenza 
Per accedere ai corsi è necessario sostenere un test VPI per via telematica, che verte sulla comprensione di un brano in lingua italiana. Il corso di laurea magistrale a ciclo unico ha durata quinquennale. 
Corsi di laurea magistrale
     \item Corso di Laurea Magistrale in Scienze e Gestione dei Servizi 
\end{itemize}
Per accedere bisogna sostenere un colloquio che riguarda conoscenze di matematica e statistica, diritto e sociologia generale. 

\subsection{Studiare in Bicocca}
Il corso di laurea a ciclo unico permette l'accesso ai concorsi pubblici per le professioni forensi (avvocato, magistrato, notaio). Il regolamento di facoltà garantisce nove appelli all'anno, senza prevedere il salto d'appello. 

\subsection{Contatti}
La segreteria didattica è situata al secondo piano dell'edificio U6. Per informazioni sulla didattica come orari delle lezioni, conferenze, e avvisi dei docenti, visitate il sito ufficiale: www.giurisprudenza.unimib.it 
Rappresentanti di facoltà: rapp\_iuris@unimib.it 

\mysec{Medicina e Chirurgia}

\subsection{Presentazione}
In università ci sono più di 2000 studenti di medicina che studiano tra sette corsi di laurea triennale, due corsi di laurea magistrale e due corsi di laurea magistrale a ciclo unico. 

\subsection{Corsi di laurea triennale}
\begin{itemize}
\item Fisioterapia (63 posti) 
\item Igiene dentale (32 posti) 
\item Infermieristica (357 posti) 
\item Ostetricia (43 posti) 
\item Tecniche di laboratorio biomedico (27 posti) 
\item Tecniche di radiologia medica, per immagini e radioterapia (32 posti) 
\item Terapia della neuro e psicomotricità dell'età evolutiva (27 posti) 
\end{itemize}

\subsection{Corsi di laurea magistrale}
\begin{itemize}
\item Biotecnologie mediche 
\item Scienze infermieristiche ed ostetriche
\end{itemize}

\subsection{Corsi di laurea magistrale a ciclo unico}
\begin{itemize}
\item Medicina e chirurgia (135 posti) 
\item Odontoiatria e protesi dentaria (20 posti) 
\end{itemize}
I corsi di laurea magistrale a ciclo unico hanno la durata di 6 anni. 

\subsection{Contatti}
Sito ufficiale di medicina: http://www.medicina.unimib.it 


\mysec{Psicologia}

\subsection{Presentazione}
Psicologia ha più di 3000 studenti e offre due corsi di laurea triennale e quattro corsi di laurea magistrale, di cui uno in collaborazione con Scienze. 

\subsection{Corsi di laurea triennale}
\begin{itemize}
\item Comunicazione e Psicologia (122 posti) 
\item Scienze e tecniche Psicologiche (500 posti) 
\end{itemize}

\subsection{Corsi di laurea magistrale}
\begin{itemize}
\item Psicologia clinica, dello sviluppo e neuropsicologia (260 posti) 
\item Psicologia dei processi sociali, decisionali e dei comportamenti economici 
\item Psicologia dello sviluppo e dei processi educativi 
\item Teoria e tecnologia della comunicazione (con Scienze)
\end{itemize}

\subsection{Studiare in Bicocca}
Il corso di laurea in Teoria e tecnologia della comunicazione è un corso interfacoltà, con contenuti che riguardano la comunicazione e l'informatica.  Per l'accesso ai corsi a numero programmato è opportuno consultare i bandi di ammissione dei corsi di laurea, sul sito di Psicologia. 

\subsection{Contatti}
Sito di Psicologia: www.psicologia.unimib.it 
Informazioni per l'iscrizione: http://www.psicologia.unimib.it/01\_ iscriversi/offerta.php 
Rappresentanti degli studenti: psicologia.rappresentanti@gmail.com 

\mysec{Scienze della Formazione}
\subsection{Presentazione}
Scienze della Formazione conta circa 6000 studenti e offre due corsi di laurea triennali, quattro magistrali, ed uno magistrale a ciclo unico. 
\subsection{Corsi di laurea triennale}

\begin{itemize}
\item Scienze dell'educazione 
\item Comunicazione interculturale
\end{itemize}
Da quest'anno il corso di laurea in scienze dell'educazione è ad accesso programmato; 690 posti saranno assegnati tramite un test nel mese di settembre. 
\subsection{Corsi di laurea magistrale}

\begin{itemize}
\item Scienze pedagogiche 
\item Scienze antropologiche ed etnologiche 
\item Formazione e sviluppo delle risorse umane 
\item Psicologia dello sviluppo e dei processi educativi 
\end{itemize}
I corsi di laurea  in Scienze pedagogiche e Formazione e sviluppo delle risorse umane prevedono un tirocinio obbligatorio. 
\subsection{Corsi di laurea magistrale a ciclo unico}
\begin{itemize}
\item Scienze della formazione primaria (400 posti) 
\end{itemize}
Il corso di laurea magistrale a ciclo unico in Scienze della formazione primaria ha dallo scorso anno (2011/12) durata quinquennale ed è a numero programmato: 400 posti, più 2 per studenti non comunitari. Il corso abilita alla professione di insegnante nelle scuole dell'infanzia e nella scuola primaria. 
\subsection{Studiare in Bicocca}
La frequenza non è obbligatoria, ma viene raccomandata anche in ragione delle facilitazioni per i frequentanti, come sgravi in termini di carico di studio oppure con l'ausilio di prove intermedie o preappelli. 
\subsection{Contatti}
Sito di facoltà: www.formazione.unimib.it 
I contatti dei rappresentanti si trovano nella sezione "persone". 
Piattaforma e-learning, per studenti iscritti: http://formazione.elearning.unimib.it/ 

\mysec{Scienze Matematiche, Fisiche e Naturali}

\subsection{Presentazione}
Scienze ha più di 5000 iscritti e offre 10 corsi di laurea triennale e 11 corsi di laurea magistrale.

\subsection{Corsi di laurea triennale}
\begin{itemize}
\item Biotecnologie (225 posti) 
\item Fisica 
\item Informatica (300)
\item Matematica 
\item Ottica e optometria (150) 
\item Scienza dei materiali 
\item Scienze biologiche (225 posti) 
\item Scienze e tecnologie chimiche (100) 
\item Scienze e tecnologie geologiche 
\item Scienze e tecnologie per l'ambiente (150)
\end{itemize}

\subsection{Corsi di laurea magistrale}
\begin{itemize}
\item Astrofisica e fisica dello spazio 
\item Biotecnologie industriali
\item Biologia
\item Fisica
\item Informatica
\item Matematica
\item Scienza dei materiali 
\item Scienze e tecnologie chimiche 
\item Scienze e tecnologie geologiche
\item Scienze e tecnologie per l'ambiente e il territorio 
\item Teoria e tecnologia della comunicazione
\end{itemize}

Il corso in Teoria e tecnologia della comunicazione è tenuto in collaborazione con Psicologia. 

\subsection{Studiare in Bicocca}
L'accesso ai corsi triennali senza il numero programmato prevede il superamento di una prova che verte principalmente sulle conoscenze di matematica e logica. Per il primo anno, la maggioranza dei corsi di laurea triennale della Scuola di Scienze è a numero programmato. I corsi di laurea magistrale prevedono requisiti curricolari e competenze che sono specificati sul manifesto dei rispettivi corsi di laurea. 
Scienze offre inoltre dei precorsi di richiami di matematica e di metodologia dello studio universitario. Durante il primo anno, sono previsti corsi di recupero per chi non avesse superato il VPI. 

\subsection{Contatti}
Sito di Scienze: www.scienze.unimib.it 

\mysec{Scienze Statistiche}

\subsection{Presentazione}
Scienze Statistiche conta circa 650 iscritti, il numero più piccolo di studenti all'interno dell'ateneo. Tutti i corsi di Laurea erogati fanno parte delle cosiddette lauree panda e pertanto è previsto per tutti gli iscritti il rimborso della tassa di iscrizione del primo anno e parte dei contributi versati. 
La sede è situata al secondo piano dell'edificio U7, dove è possibile trovare uffici dei docenti, segreteria didattica e presidenza. 

\subsection{Corsi di laurea triennale} 

\begin{itemize}
\item Corso di Laurea Triennale in Scienze Statistiche ed Economiche (SSE) 
\item Corso di Laurea Triennale in Statistica e Gestione delle Informazioni (SGI) 
\end{itemize}

\subsection{Corsi di laurea magistrale}
\begin{itemize}
\item Corso di Laurea Magistrale in Scienze Statistiche ed Economiche (CLAMSES) 
\item Corso di Laurea Magistrale in Biostatistica e Statistica Sperimentale (BIOSTAT) 
\end{itemize}

\subsection{Studiare in Bicocca}
Una laurea in Scienze Statistiche offre buone, se non ottime, possibilità di lavoro. Infatti la richiesta di statistici in Lombardia è superiore al numero dei laureati di ogni anno. 

\subsection{Contatti}
Segreteria didattica: tel. 02.6448.5828 
Ufficio orientamento e stage: tel. 02.6448.5876
Segreteria di presidenza: tel. 02.6448.5800 
I docenti di riferimento per i quattro corsi di laurea sono: 
\begin{itemize}
   \item SSE: Prof. Ongaro Andrea andrea.ongaro@unimib.it 
   \item SGI: Prof.ssa Migliorati Sonia sonia.migliorati@unimib.it 
   \item CLAMSES: Prof. Manera Matteo matteo.manera@unimib.it 
   \item BIOSTAT: Prof. Vittadini Giorgio giorgio.vittadini@unimib.it 
\end{itemize}
Per contatti scrivere a rappresentanti.studenti@statistica.unimib.it 
Per ulteriori info www.statistica.unimib.it 

\include{capitoli/capitolo5-1}
\include{capitoli/capitolo5-2}
\subsection{150 Ore}
Tra le opportunità previste dalla legge n.390 del 1991 a sostegno del diritto allo studio, l'università ha previsto le cosiddette 150 ore ovvero collaborazioni con gli studenti per lo svolgimento di diverse attività (tutor dei laboratori informatici, supporto alle biblioteche o alle Segreterie Studenti, orientamento per le matricole, etc.) per un massimo – appunto - di 150 ore. Queste collaborazioni sono a tempo parziale e retribuite 9€ all'ora. Si accede tramite bando basato su criteri di reddito e merito, per cui è necessario essere iscritti almeno al secondo anno per potervi partecipare. Durante ogni anno accademico vengono proposti diversi bandi, sia d'Ateneo sia di Facoltà, per i quali possono essere richiesti alcuni requisiti tecnici (es. abilità informatiche di base). Se siete interessati, controllate periodicamente http://www.unimib.it/go/45056 sul quale vengono pubblicati bandi e graduatorie.

\subsection{Job Placement}
Promosso dall'ufficio Job Placement, il servizio VULCANO (Vetrina Universitaria Laureati con Curricula per le Aziende Navigabile On-line) offre la possibilità a tutti gli studenti, laureati o laureandi, di essere inseriti in un database di curricula che permette all'ufficio di favorire e perseguire l'incontro tra offerta e domanda di lavoro. Gli iscritti al servizio ricevono via mail proposte inoltrate da aziende interessate ai profili professionali proposti, sia per offerte di lavoro sia per possibilità di stage.
Ogni anno, inoltre, l'Ateneo organizza il Career Day, giornata d'incontro tra studenti/laureati e aziende dove vengono proposti anche attività per la stesura di un buon curriculum e per l'introduzione al mondo del lavoro.
Per maggiori informazioni, è utile consultare la pagina web http://http://www.unimib.it/go/45763, dove sono presenti anche gli orari dello Sportello d'Orientamento.
 
\subsection{Counselling psicologico}
La Bicocca offre anche un utile, ma spesso sconosciuto, servizio di counselling per chi avesse problemi di studio o situazioni personali che inibiscono il corretto svolgimento della carriera universitaria. Il servizio consiste in un ciclo medio-breve di incontri individuali con uno psicoterapeuta od uno psicologo clinico specializzati ad operare con pazienti in età tardo- adolescenziale e giovane-adulta. Durante il percorso lo specialista cercherà di stabilire con lo studente degli obiettivi chiari da raggiungere entro la fine della serie di incontri.
Il servizio è totalmente gratuito, per prenotare un incontro o semplicemente avere informazioni è possibile inviare un' e-mail o prendere contatto telefonicamente. I contatti sono disponibili all'indirizzo http://www.unimib.it/go/46063.

\subsection{Sport}
Correre per arrivare in università o per prendere il treno all'ultimo non si può considerare propriamente uno sport. Meglio qualcosa di classico, divertente e, possibilmente, a buon prezzo.
Per questo potete rivolgervi al CUS (Centro Universitario Sportivo), che offre corsi a prezzi convenienti, fornisce sconti ed agevolazioni per alcuni impianti comunali e organizza settimane bianche, gite ed escursioni. In Bicocca c'è una palestra convenzionata (in U12), accessibile a studenti e dipendenti, aperta dal lunedì a venerdì dalle ore 12 alle ore 20 (dalle ore 10 alle ore 12 solo per i residenti del pensionato) che ha dei prezzi veramente imbattibili! Per informazioni www.cusmilano.it, su Facebook http://www.facebook.com/cusbicocca, oppure rivolgetevi al CUS point in U6 al primo piano.

\subsection{Biblioteche e aree studio}
La Biblioteca d'Ateneo ha un'unica gestione ma è strutturata in tre sedi (quella Centrale, di Scienze e di Medicina). La biblioteca è il luogo ideale per studiare in tranquillità e che offre molti servizi: consultazione, prestito, prestito interbibliotecario, fotocopie, reperimento di articoli di vari natura, consulenza bibliografica, spazi di studio individuali per chi è sotto tesi e poi vi è un intuitivo catalogo on line (OPAC).
Le tre sedi sono:
\begin{itemize}
\item Sede Centrale: Piazza dell'Ateneo Nuovo 1, edificio U6 II piano. Aree disciplinari: diritto, economia, informatica, psicologia, sociologia, scienze della formazione e statistica. Orario di apertura: dal lunedì al giovedì dalle 9.00 alle 19.30 e il venerdì dalle 9.00 alle 18.30.
\item Sede di Scienze: Piazza della Scienza 3, edificio U2 I piano (sala monografie, dal lunedì al venerdì, dalle 9.00 alle 18.30) e piano -1 (sala periodici, aperta fino alle 16.00). Aree disciplinari: matematica, fisica, biologia, chimica e geologia.
\item Sede di Medicina: Via Cadore 48, Monza, edificio U8 piano terra. Aree disciplinari: medicina. Orario di apertura: dal lunedì al venerdì, dalle 9.00 alle 18.30.
\end{itemize}
Inoltre nel campus della Bicocca è presente un'altra Biblioteca afferente al CIDiS, che si trova al secondo piano dell'edificio U12. A differenza della biblioteca d'Ateneo è gestita da un organo interunivesitario (appunto il CIDiS) . Per poter usufruire del prestito dei libri è necessario pagare una quota di 10 euro. Grazie al successo della sperimentazione degli anni scorsi sull'apertura serale, la biblioteca del CIDiS ha confermato gli orari di apertura dal lunedì al venerdì dalle 9.00 alle 22.00 ed ha esteso i suoi servizi alla serata del sabato dalle 18.00 alle 22.00.
Altri luoghi di studio sono le Aree Studio sparse tra i vari edifici dell'Università, dove si può trovare una maggiore tranquillità rispetto ai tavoli nei corridoi o nei cortili interni.

\subsection{Copisterie}
Quasi certamente nella vostra carriera universitaria vi troverete con la necessità di stampare dispense in media da 200 pagine, fotocopiare una lezione persa oppure un intero quaderno di appunti di una vostra compagna, per non parlare poi della tesi. Le principali copisterie nelle vicinanze di Piazza della Scienza e di Piazza dell'Ateneo Nuovo sono:
\begin{itemize}
\item Copisteria al n°7: via Luigi Pulci, 7;
\item All.net: p.zza della Trivulziana, 2;
\item Digicopy: viale Sarca, 173;
\item Centro Copie Bicocca: viale Sarca, 198.
\end{itemize}

\mysec{Le altre associazioni}

\subsection{Associazione Studenti Bicocca}
L'Associazione Studenti Bicocca (ASB) è un'associazione culturale, senza fini di lucro, nata nel 2002 dalla collaborazione di studenti provenienti da diverse facoltà dell'Università degli Studi di Milano Bicocca. ASB è quindi un punto di incontro di studenti di diverse facoltà, dove si ha modo sia di socializzare che promuovere concretamente i propri interessi, come conferenze, feste, concorsi fotografici, aperitivi culturali, giornate di sensibilizzazione sulla sostenibilità. ASB cura altresì il forum degli studenti della Bicocca (www.studentibicocca.org), con più di 38.000 iscritti e 5.000 accessi giornalieri, dove i ragazzi hanno modo di confrontarsi sui più disparati temi nonché sulle materie di esame, trovando anche materiali gratuiti da scaricare, condivisi da altri studenti. Per informazioni scrivete ad asb@studentibicocca.org o visitate www.studentibicocca.org.

\subsection{ESN Bicocca}
Erasmus Student Network (ESN) è una delle più grandi associazioni di studenti in Europa, fondata nel 1989 per supportare e incrementare i programmi di mobilità studentesca.\\
ESN Milano-Bicocca è l'associazione che si occupa di aiutare e assistere gli studenti Erasmus ed Exchange all'interno del nostro Ateneo, attraverso tantissime attività diverse, che vanno dall'accoglienza e all'orientamento ai corsi universitari, all'organizzazione di attività culturali e ricreative come viaggi, feste e incontri tra studenti italiani e stranieri. 
Per informazioni, il loro sito è http://www.esnbicocca.it/bicocca/

\mysec{Mobilità internazionale}
\subsection{Erasmus: studiare all'estero}
Anzitutto, che cos'è il Progetto Erasmus? L'Erasmus è il principale progetto europeo di mobilità studentesca internazionale e permette ogni anno a migliaia di studenti di muoversi per un periodo di tempo (dai tre mesi ad un anno) ed andare a studiare e dare esami in un altro paese europeo, oppure svolgere uno stage lavorativo.\\
Lo studente in Erasmus per studio è equiparato agli studenti dell'università ospitante e avrà quindi accesso a tutti i servizi offerti loro, oltre ad alcuni servizi specifici come l'aiuto nella ricerca dell'alloggio o i corsi di lingua. Per partire è necessario consultare i bandi che ogni anno la propria facoltà pubblica, solitamente in febbraio/marzo dell'anno precedente a quello che vi interessa, e fare domanda attraverso gli appositi uffici (la procedura completa è descritta all'interno del bando di riferimento). \\
Ogni facoltà propone le destinazioni con cui è convenzionata.\\
Possono partire tutti gli studenti iscritti ad un anno successivo al primo (il primo anno si può fare domanda per partire il secondo). Prima di partire è necessario stipulare un learning agreement, ovvero un elenco degli esami che si intendono sostenere all'estero, per avere la certezza, al ritorno, che tutti gli esami sostenuti siano riconosciuti e quali saranno, per esempio, i crediti e il voto attribuiti ad ognuno (non tutti i paesi usano sistemi con crediti e voti in trentesimi).\\
Lo studente in Erasmus per studio riceve un contributo mensile per coprire parte delle spese del soggiorno all'estero. Esso è erogato dall'Ateneo su finanziamenti europei e può essere di 230€ o di 280€ per ogni mese di permanenza all'estero, in base al costo della vita nel Paese dell'università di destinazione. Ad esso va sommata un'integrazione che la Bicocca fornisce con fondi propri, la quale viene decisa ogni anno e la cui entità dipende dall'ISEEU dello studente.\\
A tal proposito, lo studente che intende andare in Erasmus per studio dovrebbe ricordarsi di presentare la dichiarazione ISEEU a settembre dell'anno accademico in cui partecipa al bando (quindi l'anno prima di quello in cui effettivamente parte). In caso contrario non potrà ricevere l'integrazione.\\
Anche il CIDiS (Consorzio Interuniversitario per il Diritto allo Studio) bandisce delle borse per gli studenti in partenza per università estere assegnate in base al reddito e ai punti di credito conseguiti negli anni.\\
Gli studenti interessati ad un'esperienza di stage all'estero devono invece partecipare ai bandi di Erasmus Placement. In questo caso la scelta della destinazione non è vincolata ad accordi tra la Bicocca e altre università ma dipende dall'ente in cui si intende svolgere lo stage, che può essere un'azienda o un centro di ricerca e formazione. La ricerca di un ente ospite spetta allo studente, il quale però può provare a chiedere suggerimenti ai propri professori.\\
La borsa di studio per l'Erasmus Palcement ammonta a 500€ al mese.\\
Quante volte si può andare in Erasmus? Da quest'anno, con il nuovo programma Erasmus+, ogni studente ha di diritto fino a 12 mesi di permanenza all'estero, anche non consecutivi, per ciclo di studi. Questo vuol dire che triennale e magistrale vengono contate separatamente: passando alla magistrale i mesi a disposizione tornano ad essere 12!

\subsubsection{Lingue}
Se si va in Erasmus per studio, l'università di partenza può richiedere la conoscenza della lingua del paese di destinazione che avete scelto, possibilmente certificata da un diploma. Altrimenti è comunque sufficiente aver passato il test di conoscenze linguistiche di inglese (o della lingua del paese di destinazione) di Ateneo. Di solito, inoltre, per le lingue meno conosciute, le Università di destinazione organizzano dei corsi specifici destinati agli studenti Erasmus.\\
Da quest'anno, tuttavia, la Bicocca richiede una competenza linguistica di inglese o della lingua dell'ateneo ospite, se è tra le lingue europee più parlate, almeno di livello B2 (quindi più alta di quella del test iniziale di ateneo) prima di acconsentire alla partenza. Chi non dispone di tale certificato dovrà seguire dei corsi di lingua e superare un test finale che si terrà a Luglio.


\subsection{Altri programmi di scambio}
La Bicocca offre poi altri programmi di scambio:

\begin{itemize}
\item Erasmus Mundus: progetto che offre la possibilità di studiare all'estero, non durante il proprio percorso accademico curricolare, bensì attraverso un master. 

\item Extra: progetto che consente lo svolgimento di un periodo di studio all'estero finalizzato alla preparazione della tesi di Laurea Specialistica/Magistrale presso università o centri di ricerca con i quali siano già attivi contatti e/o iniziative di collaborazione accademica o scientifica con l'Università Bicocca. Nell'ambito di tale Programma, sono stati finanziati dei premi di studio da parte della Fondazione Cariplo dell'importo mensile lordo pari a 750€.\\
Nota bene: da quest'anno si può accedere al bando Extra solo se si ha già avuto un'esperienza di studio all'estero. Quindi se vuoi parteciparvi ti conviene pianificare di andare in Erasmus l'anno prima, o anche l'anno stesso.

\item Exchange: programma che da la possibilità di studiare e dare esami in un paese europeo o extra europeo presso uno dei Partners Exchange di Ateneo. Si può partecipare anche se si ha già esaurito i mesi Erasmus.

\item Summer School: le Summer Schools offrono la possibilità di andare in un'università, estera o anche italiana, per frequentare corsi estivi di approfondimento su tematiche o settori del proprio corso di laurea, della durata di alcune settimane. I corsi seguiti durante la Summer School possono essere riconosciuti come CFU a scelta, ma solo su richiesta dello studente e senza che debba esserci un accordo tra l'università di appartenenza e quella ospite.\\
Spesso i bandi per le Summer School sono disponibili sui siti delle facoltà che le organizzano, piuttosto che su quello di ateneo.
\end{itemize}


\subsection{Per maggiori informazioni}
Per maggiori informazioni, potete consultare il sito internet della Bicocca all'indirizzo http://www.unimib.it/go/45776/Home/Italiano/Menu-sinistra/Internazionalizzazione/Mobilita-internazionale \\
Per un contatto diretto con l'ufficio della Bicocca che si occupa degli Erasmus, potete scrivere a: international.office@unimib.it

\include{capitoli/capitolo6}
\mysec{Contatti}
Puoi trovare le aule dei rappresentanti degli studenti  nell'edificio U2 (Piazza della Scienza - Fisica) al piano terra, tel. 02.6448 2061 oppure nell'edificio U6 (Piazza dell'Ateneo NuovoEconomia,Psicologia, Scienze della Formazione, Giurisprudenza, Rettorato) al piano -1, tel. 02.6448 6992
Se vuoi contattarci puoi farlo anche via e-mail all’indirizzo listedisinistra@gmail.com oppure visitando il sito www.ListediSinistra.org dove ti puoi iscrivere alla nostra newsletter utilizzata per aggiornamenti riguardo iniziative e incontri o per ricordare scadenze accademiche e amministrative.
Da quest'anno siamo anche su facebook all'indirizzo http://www.facebook.com/ListediSinistra 
Non ti basta? Se hai un problema relativo ad uno specifico Corso di Studi, ecco qui a chi puoi rivolgerti:

Scienze:
\begin{itemize}
\item Matematica: n.grittini@campus.unimib.it (Nicola Grittini)
\item Biotecnologie: m.gnugnoli@campus.unimib.it (Marco Gnugnoli)
\item Informatica: rappresentantistudenti@disco.unimib.it
\item Fisica: e.panontin@campus.uimib.it (Enrico Panontin)
\item Scienze e tecnologie per l'ambiente: s.calabrese4@campus.unimib.it (Silvia Calabrese)
\end{itemize}   
Sociologia: 
\begin{itemize}
\item t.pettinato@campus.unimib.it (Thomas Pettinato) oppure rappresentanti.sociologia@gmail.com
\end{itemize}
Giurisprudenza:
\begin{itemize}
\item s.zanco@campus.unimib.it (Sebastiano Zanco)
\end{itemize}   
Medicina: 
\begin{itemize}
\item d.celsi@campus.unimib.it (Davide Celsi)
\end{itemize}   
Statistica: 
\begin{itemize}
\item a.torti7@gmail.com (Andrea Torti)
\end{itemize}
Economia: 
\begin{itemize}
\item Ecofin: g.papaleo92@gmail.com (Giuseppe Papaleo)
\end{itemize}

\end{multicols}
%\printindex

\end{document}

