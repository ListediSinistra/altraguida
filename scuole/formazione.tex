\mysec{Scienze della Formazione}
\subsection{Presentazione}
Scienze della Formazione conta circa 6000 studenti e offre due corsi di laurea triennali, quattro magistrali, ed uno magistrale a ciclo unico. 
\subsection{Corsi di laurea triennale}

\begin{itemize}
\item Scienze dell'educazione 
\item Comunicazione interculturale
\end{itemize}
Da quest'anno il corso di laurea in scienze dell'educazione è ad accesso programmato; 690 posti saranno assegnati tramite un test nel mese di settembre. 
\subsection{Corsi di laurea magistrale}

\begin{itemize}
\item Scienze pedagogiche 
\item Scienze antropologiche ed etnologiche 
\item Formazione e sviluppo delle risorse umane 
\item Psicologia dello sviluppo e dei processi educativi 
\end{itemize}
I corsi di laurea  in Scienze pedagogiche e Formazione e sviluppo delle risorse umane prevedono un tirocinio obbligatorio. 
\subsection{Corsi di laurea magistrale a ciclo unico}
\begin{itemize}
\item Scienze della formazione primaria (400 posti) 
\end{itemize}
Il corso di laurea magistrale a ciclo unico in Scienze della formazione primaria ha dallo scorso anno (2011/12) durata quinquennale ed è a numero programmato: 400 posti, più 2 per studenti non comunitari. Il corso abilita alla professione di insegnante nelle scuole dell'infanzia e nella scuola primaria. 
\subsection{Studiare in Bicocca}
La frequenza non è obbligatoria, ma viene raccomandata anche in ragione delle facilitazioni per i frequentanti, come sgravi in termini di carico di studio oppure con l'ausilio di prove intermedie o preappelli. 
\subsection{Contatti}
Sito di facoltà: www.formazione.unimib.it 
I contatti dei rappresentanti si trovano nella sezione "persone". 
Piattaforma e-learning, per studenti iscritti: http://formazione.elearning.unimib.it/ 
