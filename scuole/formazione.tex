Scienze della Formazione
Presentazione
Scienze della Formazione conta circa 6000 studenti e offre due corsi di laurea triennali, quattro magistrali, ed uno magistrale a ciclo unico. 
Corsi di laurea triennale
     • Scienze dell'educazione 
     • Comunicazione interculturale 
Da quest'anno il corso di laurea in scienze dell'educazione è ad accesso programmato; 690 posti saranno assegnati tramite un test nel mese di settembre. 
Corsi di laurea magistrale
     • Scienze pedagogiche 
     • Scienze antropologiche ed etnologiche 
     • Formazione e sviluppo delle risorse umane 
     • Psicologia dello sviluppo e dei processi educativi 
I corsi di laurea  in Scienze pedagogiche e Formazione e sviluppo delle risorse umane prevedono un tirocinio obbligatorio. 
Corsi di laurea magistrale a ciclo unico
     • Scienze della formazione primaria (400 posti) 
Il corso di laurea magistrale a ciclo unico in Scienze della formazione primaria ha dallo scorso anno (2011/12) durata quinquennale ed è a numero programmato: 400 posti, più 2 per studenti non comunitari. Il corso abilita alla professione di insegnante nelle scuole dell'infanzia e nella scuola primaria. 
Studiare in Bicocca
La frequenza non è obbligatoria, ma viene raccomandata anche in ragione delle facilitazioni per i frequentanti, come sgravi in termini di carico di studio oppure con l'ausilio di prove intermedie o preappelli. 
Contatti
Sito di facoltà: www.formazione.unimib.it 
I contatti dei rappresentanti si trovano nella sezione "persone". 
Piattaforma e-learning, per studenti iscritti: http://formazione.elearning.unimib.it/ 
