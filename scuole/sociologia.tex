\mysec{Sociologia}

\subsection{Presentazione}
Sociologia offre quattro corsi di laurea triennale e quattro corsi di laurea magistrale, di cui uno interfacoltà. La Scuola è inoltre da vari anni al primo posto in Italia, secondo la Guida Università (Repubblica - Censis). 

\subsection{Corsi di laurea triennale}
\begin{itemize}
\item Scienze del turismo e comunità locale (180 posti) (*) 
\item Scienze dell'organizzazione 
\item Servizio sociale (120 posti) (*) 
\item Sociologia
\end{itemize}
(*) Corsi a numero programmato che prevedono dei posti aggiuntivi per studenti non comunitari (4 per Scienze del turismo e comunità locale, di cui 2 cittadini della Repubblica Popolare Cinese, e 6 per Servizio Sociale, di cui 4 per cittadini cinesi). 
L'accesso a questi Corsi a Numero Programmato (NP) è subordinato al superamento di una prova scritta (2 Settembre 2013 Servizio Sociale; 16 Settembre 2013 Scienze del turismo e comunità locale). 
I restanti corsi prevedono un test della Valutazione della Preparazione Iniziale (VPI) il cui esito non nuoce all'iscrizione, ma obbliga la frequenza di determinati corsi aggiuntivi. Il VPI  verte su conoscenze generali, storiche, comprensione della lingua italiana ed esercizi di logica.
Le lezioni del corso triennale in Scienze dell'organizzazione, contrariamente agli altri corsi della Scuola, si tengono presso l'ospedale vecchio di Monza. La sede è facilmente raggiungibile dalla stazione di Monza. 

\subsection{Corsi di laurea magistrale}
\begin{itemize}
\item Programmazione e gestione delle politiche e dei servizi sociali (PROGEST) (80 posti) (*) 
\item Scienze e gestione dei servizi 
\item Sociologia 
\item Turismo, territorio e sviluppo locale
\end{itemize}
(*) Il corso prevede 4 posti aggiuntivi per studenti non comunitari, di cui 1 per cittadini cinesi. 
Sociologia e Turismo, territorio e sviluppo locale richiedono l'invio di una "domanda di valutazione" per verificare l'incontro con criteri di merito necessari per l'accettazione della domanda di iscrizione.

\subsection{Contatti}
Sito di facoltà: www.sociologia.unimib.it 
(Manifesto annuale degli studi: http://is.gd/fk5odj)
