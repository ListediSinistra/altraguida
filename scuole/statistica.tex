Scienze Statistiche

Presentazione
Scienze Statistiche conta circa 650 iscritti, il numero più piccolo di studenti all'interno dell'ateneo. Tutti i corsi di Laurea erogati fanno parte delle cosiddette lauree panda e pertanto è previsto per tutti gli iscritti il rimborso della tassa di iscrizione del primo anno e parte dei contributi versati. 
La sede è situata al secondo piano dell'edificio U7, dove è possibile trovare uffici dei docenti, segreteria didattica e presidenza. 

Corsi di laurea triennale 
    • Corso di Laurea Triennale in Scienze Statistiche ed Economiche (SSE) 
    • Corso di Laurea Triennale in Statistica e Gestione delle Informazioni (SGI) 

Corsi di laurea magistrale
    • Corso di Laurea Magistrale in Scienze Statistiche ed Economiche (CLAMSES) 
    • Corso di Laurea Magistrale in Biostatistica e Statistica Sperimentale (BIOSTAT) 

Studiare in Bicocca
Una laurea in Scienze Statistiche offre buone, se non ottime, possibilità di lavoro. Infatti la richiesta di statistici in Lombardia è superiore al numero dei laureati di ogni anno. 

Contatti
Segreteria didattica: tel. 02.6448.5828 
Ufficio orientamento e stage: tel. 02.6448.5876
Segreteria di presidenza: tel. 02.6448.5800 
I docenti di riferimento per i quattro corsi di laurea sono: 
       • SSE: Prof. Ongaro Andrea andrea.ongaro@unimib.it 
       • SGI: Prof.ssa Migliorati Sonia sonia.migliorati@unimib.it 
       • CLAMSES: Prof. Manera Matteo matteo.manera@unimib.it 
       • BIOSTAT: Prof. Vittadini Giorgio giorgio.vittadini@unimib.it 
Per contatti scrivere a rappresentanti.studenti@statistica.unimib.it 
Per ulteriori info www.statistica.unimib.it 
