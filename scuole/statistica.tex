\mysec{Statistica}

\subsection{Presentazione}
Scienze Statistiche conta circa 650 iscritti. Tutti i corsi di Laurea erogati fanno parte delle cosiddette lauree panda e pertanto è previsto per tutti gli iscritti il rimborso della tassa di iscrizione del primo anno e parte dei contributi versati proporzionalmente rispetto a media e cfu ottenuti. 
La sede è situata al secondo piano dell'edificio U7, dove è possibile trovare uffici dei docenti e segreteria didattica.
 
\subsection{Corsi di laurea triennale} 

\begin{itemize}
\item Corso di Laurea Triennale in Scienze Statistiche ed Economiche (SSE) 
\item Corso di Laurea Triennale in Statistica e Gestione delle Informazioni (SGI) 
\end{itemize}

\subsection{Corsi di laurea magistrale}
\begin{itemize}
\item Corso di Laurea Magistrale in Scienze Statistiche ed Economiche (CLAMSES) 
\item Corso di Laurea Magistrale in Biostatistica e Statistica Sperimentale (BIOSTAT)
\item Corso di Laurea Magistrale in Scienze e Gestione dei Servizi (MAGES)
\end{itemize}

\subsection{Studiare in Bicocca}
Una laurea in Scienze Statistiche offre buone, se non ottime, possibilità di lavoro. Infatti la richiesta di statistici in Lombardia è superiore al numero dei laureati di ogni anno. 

\subsection{Contatti}
Segreteria didattica: tel. 02.6448.5811
Ufficio orientamento e stage: tel. 02.6448.5876
I docenti di riferimento per i quattro corsi di laurea sono: 
\begin{itemize}
\item SSE: Prof. Ongaro Andrea andrea.ongaro@unimib.it 
\item SGI: Prof. Giancarlo Travaglini giancarlo.travaglini@unimib.it 
\item CLAMSES: Prof. Manera Matteo matteo.manera@unimib.it 
\item BIOSTAT: Prof. Antonella Zambon antonella.zanbon@unimib.it 
\item MAGES: Prof. Dario Cavenago dario.cavenago@unimib.it
\end{itemize}
Per ulteriori info www.statistica.unimib.it 
