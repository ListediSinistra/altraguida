\mysec{Giurisprudenza}

\subsection{Presentazione}
L'università offre un corso di laurea triennale, un corso di laurea magistrale e un corso di laurea magistrale a ciclo unico, tutti ad accesso libero nell'ambito delle scienze giuridiche.
Corsi di laurea triennale
\begin{itemize}
     \item Corso di Laurea Triennale in Scienze dei Servizi Giuridici 
Corsi di laurea magistrale a ciclo unico
     \item Corso di Laurea Magistrale a ciclo unico in Giurisprudenza 
Per accedere ai corsi è necessario sostenere un test VPI per via telematica, che verte sulla comprensione di un brano in lingua italiana. Il corso di laurea magistrale a ciclo unico ha durata quinquennale. 
Corsi di laurea magistrale
     \item Corso di Laurea Magistrale in Scienze e Gestione dei Servizi 
\end{itemize}
Per accedere bisogna sostenere un colloquio che riguarda conoscenze di matematica e statistica, diritto e sociologia generale. 

\subsection{Studiare in Bicocca}
Il corso di laurea a ciclo unico permette l'accesso ai concorsi pubblici per le professioni forensi (avvocato, magistrato, notaio). Il regolamento di facoltà garantisce nove appelli all'anno, senza prevedere il salto d'appello. 

\subsection{Contatti}
La segreteria didattica è situata al secondo piano dell'edificio U6. Per informazioni sulla didattica come orari delle lezioni, conferenze, e avvisi dei docenti, visitate il sito ufficiale: www.giurisprudenza.unimib.it 
Rappresentanti di facoltà: rapp\_iuris@unimib.it 
