\mysec{Scienze Matematiche, Fisiche e Naturali}

\subsection{Presentazione}
Scienze ha più di 5000 iscritti e offre 10 corsi di laurea triennale e 11 corsi di laurea magistrale.

\subsection{Corsi di laurea triennale}
\begin{itemize}
\item Biotecnologie (225 posti) 
\item Fisica 
\item Informatica (300)
\item Matematica 
\item Ottica e optometria (150) 
\item Scienza dei materiali 
\item Scienze biologiche (225 posti) 
\item Scienze e tecnologie chimiche (100) 
\item Scienze e tecnologie geologiche 
\item Scienze e tecnologie per l'ambiente (150)
\end{itemize}

\subsection{Corsi di laurea magistrale}
\begin{itemize}
 \item Astrofisica e fisica dello spazio 
 \item Biotecnologie industriali
 \item Biologia
 \item Fisica
 \item Informatica
 \item Matematica
 \item Scienza dei materiali 
 \item Scienze e tecnologie chimiche 
 \item Scienze e tecnologie geologiche
 \item Scienze e tecnologie per l'ambiente e il territorio 
 \item Teoria e tecnologia della comunicazione
\end{itemize}

Il corso in Teoria e tecnologia della comunicazione è tenuto in collaborazione con Psicologia. 

\subsection{Studiare in Bicocca}
L'accesso ai corsi triennali senza il numero programmato prevede il superamento di una prova che verte principalmente sulle conoscenze di matematica e logica. Per il primo anno, la maggioranza dei corsi di laurea triennale della Scuola di Scienze è a numero programmato. I corsi di laurea magistrale prevedono requisiti curricolari e competenze che sono specificati sul manifesto dei rispettivi corsi di laurea. 
Scienze offre inoltre dei precorsi di richiami di matematica e di metodologia dello studio universitario. Durante il primo anno, sono previsti corsi di recupero per chi non avesse superato il VPI. 

\subsection{Contatti}
Sito di Scienze: www.scienze.unimib.it 
Precorsi: http://www.scienze.unimib.it/?page\_id=243
