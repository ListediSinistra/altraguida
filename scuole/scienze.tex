Scienze Matematiche, Fisiche e Naturali

Presentazione
Scienze ha più di 5000 iscritti e offre 10 corsi di laurea triennale e 11 corsi di laurea magistrale, incluso uno in collaborazione con Psicologia. 

Corsi di laurea triennale
     • Biotecnologie (300 posti) 
     • Fisica 
     • Informatica 
     • Matematica 
     • Ottica e optometria 
     • Scienza dei materiali 
     • Scienze biologiche (300 posti) 
     • Scienze e tecnologie chimiche 
     • Scienze e tecnologie geologiche 
     • Scienze e tecnologie per l'ambiente 

Corsi di laurea magistrale
     • Astrofisica e fisica dello spazio 
     • Biotecnologie industriali 
     • Biologia 
     • Fisica 
     • Informatica 
     • Matematica 
     • Scienza dei materiali 
     • Scienze e tecnologie chimiche 
     • Scienze e tecnologie geologiche 
     • Scienze e tecnologie per l'ambiente e il territorio 
     • Teoria e tecnologia della comunicazione 

Il corso in Teoria e tecnologia della comunicazione è tenuto in collaborazione con Psicologia. 

Studiare in Bicocca
L'accesso ai corsi triennali prevede il superamento di una prova che verte principalmente sulle conoscenze di matematica e logica. I corsi di Biotecnologie e Scienze biologiche sono per il secondo anno corsi a numero programmato. I corsi di laurea magistrale prevedono requisiti curricolari e competenze che sono specificati sul manifesto dei rispettivi corsi di laurea. 
Scienze offre inoltre dei precorsi di richiami di matematica e di metodologia dello studio universitario. Durante il primo anno, sono previsti corsi di recupero per chi non avesse superato la VPI. 

Lauree panda  
L'iscrizione ai corsi di Scienza dei materiali, Scienze e tecnologie chimiche, Fisica, Ottica e optometria e Matematica consente allo studente di ottenere al secondo anno di corso, tenuto conto del merito e del reddito, un rimborso della tassa di iscrizione e di parte dei contributi versati. 

Contatti
Sito di Scienze: www.scienze.unimib.it 
Precorsi: www.scienze.unimib.it/?main\_ page=precorsi 
